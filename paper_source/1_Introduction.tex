\section{Introduction}
\label{sec:Introduction}

Driven by ambitious environmental goals and the promising potential for improved efficiency and automation by employing the Internet of Things (IoT) and Artificial Intelligence (AI) technologies, the building sector is currently undergoing fast-paced digitalization with increasing adoption of large sensing and metering networks. While the utilization of such data offers many opportunities, it also presents great challenges to status quo data management, building operation, and building energy modeling. As a consequence, the demand is steadily growing for more flexible and scalable data and energy models that integrate seamlessly. 

As part of the digitalization of buildings, Building Information Models (BIM) have for the past two decades played a major role, unlocking significant cost savings and improving efficiency by easing communication and information exchange between contractors in the design and construction phases of buildings \cite{ullah2019}. However, despite its demonstrated success, the BIM technology is now facing significant challenges as attempts are now made to extend it beyond its intended scope to fulfill the demands of IoT-integrated smart buildings. Boje et al. \cite{BOJE2020103179}, argued that BIM is not suitable for IoT integration due to the static legacy formats and standards used to represent data. Instead, they highlighted the emerging concept of Digital Twins (DT) in buildings as a promising solution for solving these interoperability issues. 


The DT concept was first used by the National Aeronautics and Space Administration (NASA), which defined it as a \textit{"comprehensive multi-physical, multi-scale, probabilistic simulation system for vehicles or systems"} \cite{Glaessgen2012}. Following, the DT concept has been adapted in various other scientific and engineering fields such as product manufacturing, medical sciences, and smart cities \cite{Guo2022}. Although DTs as a concept is starting to emerge in the building sector, the concept is still in its infancy with no clear consensus on a common definition and which services it should provide. 


% The work presented in this paper, is developed as part of the ongoing research project \textit{Twin4Build: A holistic Digital Twin platform for decision-making support over the whole building life cycle}. The project will in collaboration with industrial partners develop and implement a digital twin solution, 


\begin{figure*}[t!]
    \centering
    \includegraphics[width=1\linewidth, trim={0 0cm 0 2cm}, clip]{DT framework system diagram.png}
    \caption{The DT concept illustrated with typical systems and components present in buildings with their virtual counterpart and the bidirectional flow of data.}
    \label{fig:DT_concept}
\end{figure*}

In this work, we adopt the definition provided by Boje et al. \cite{BOJE2020103179} and Grieves \cite{Grieves2015}. That is, a DT refers to a concept with three main constituents; a physical system, a virtual system, and a flow of data linking these two systems. This is conceptualized in the context of a generic building energy system as shown in \autoref{fig:DT_concept}. Here, the physical system is the actual asset to be managed, i.e. a building including all systems, devices, spaces, occupants, etc. The virtual system employs relevant simulation models to emulate the behavior of the physical system as closely as possible. The physical and virtual systems are linked through a bidirectional data flow, where the virtual system receives raw data in the form of static design and topology information as well as dynamic sensor and meter data. The virtual system then processes this information to adapt, monitor, simulate, and optimize based on the employed simulation models. This processed information is then communicated back to the physical system for facility management. As a basis for the work presented in this paper and the envisioned DT, three fundamental services, targeting the operational phase of the building life cycle, have been defined as follows:




\begin{enumerate}[align=left]
    \item[\textbf{Service 1}] Efficient collection and management of data generated by sensors, meters, and IoT devices through a user-friendly open-standard context information model
    \item[\textbf{Service 2}] Automated continuous commissioning and performance monitoring in real-time to detect faults and malfunctions, ensuring a smarter and more cost-effective facility management
    \item[\textbf{Service 3}] Operational strategy planning support to enable more informed decision-making by running different control and management scenarios in a zero-risk virtual environment 
\end{enumerate}





In recent years, successful implementations of fault detection and performance monitoring, as listed in \textbf{Service 2}, have been reported numerous times with different approaches. Typically, fault detection methods are divided into two categories. The first category is model-based methods, which rely on comparisons between the simulated behavior of a system with actual measurements collected from the physical unit. The second category is model-free methods, which typically utilize only operational data collected from a system to find patterns that distinguish faulty from normal operation. Although model-free methods require no models to implement, their effectiveness is also often very limited in comparison to model-based methods when applied to large-scale distributed systems such as Heating Ventilation Air-Conditioning (HVAC) systems \cite{xiao2009a}. Therefore, most efforts in fault detection and continuous commissioning of buildings have considered model-based methods, often with the use of white-box simulation tools \cite{wang2013a, oneill2014a, markoska2016a, jradi2018a}. In addition, an accurate dynamic energy model enables simulation and optimization of future operational scenarios, as required by \textbf{Service 3}, aiding decision-makers with operating the building on a day-to-day basis, taking into account weather forecasts, expected occupancy, and price signals. This potentially increases indoor comfort, reduces operational costs, and improves the overall energy flexibility of the building. 


The added value of the outlined services is well-established and has been demonstrated numerous times in case-studies \cite{Metallidou2020,Bynum2008,GUNAY2019164}. However, using traditional energy modeling tools and being challenged by various interoperability issues, such implementations usually result in highly specialized solutions tailored to each specific building. This building-by-building approach, relies on manual workflows, is costly, hard to scale, and acts as a major barrier to broadly implementing such systems. 

Therefore, inspired by the added value provided in different domains, there is a need to promote and develop DT technology for the building sector to harness the benefits provided by advanced sensing and metering devices and provide the outlined services at scale. As a key milestone in such promotion, an automated and adaptable framework for energy modeling of buildings is needed to effectively integrate such models into a building DT. 


In this work, an innovative framework is proposed and presented for automated and adaptable energy model development to provide the simulation models required by building DTs. The framework builds upon the Smart Applications REFerence (SAREF) ontology \cite{saref} to ensure interoperability by using existing classes, concepts, and relations. A selection of SAREF classes is extended with adaptable data-driven grey-box and black-box models, to describe their dynamic behavior. Furthermore, a generic framework based on the SAREF4SYST extension is presented to describe how these component models interact based on simple topology information. To provide a proof-of-concept, a demonstration case is considered to illustrate the framework implementation considering the simple single-zone system shown in \autoref{fig:DT_concept}. 


The presented framework in this work is capable to serve as a basis for the development and implementation of scalable DTs for building applications, allowing the delivery of major services to enhance the energy efficiency and intelligence quotient of buildings. This includes performance monitoring, data management, performance optimization, and strategy planning.

As established, one of the core elements of DTs is accurate simulation models. Therefore, in the following section, an overview and discussion of different approaches for building energy modeling are provided. 

\section{Building Energy Modelling}

Over the years, considerable effort has been invested in developing large-scale energy simulation tools, e.g. EnergyPlus, DOE-2, and TRNSYS \cite{crawley2001a, winkelmann-a, klein2007a}. Models developed in such tools fall under the category of white-box models, which are based entirely on first-principles building physics, typically in the form of mass and energy balance equations, which often lead to large systems of Ordinary Differential Equations (ODE). However, being the case for the majority of large-scale white-box models, they require extensive information on the building and a substantial amount of resources and time to develop. Taking EnergyPlus as an example, a typical workflow consists of three time-consuming phases before an accurate energy model is obtained. First, the building geometry is defined with detailed geometry data from floor plans, cross-sections, 3D models, etc. Second, HVAC systems design data must be specified along with material and thermal properties of the envelope and occupancy, lighting, and equipment schedules of the spaces. Finally, to achieve acceptable prediction accuracy, the model must be calibrated using actual consumption data from the building, a process that often relies on a manual trial-and-error approach that requires domain-specific expertise from the modeler \cite{reddy2006a}. In addition to the high amount of resources and time that must be dedicated to the manufacturing of these high-resolution white-box models, high computational costs and execution time is also a well-known challenge, which makes them unsuitable for operational optimization purposes \cite{reynolds2018a, chen2022a}. 

Therefore, to reduce the amount of manual work of white-box model development, considerable efforts have been put into the automatic translation of the design and geometrical data available in BIM to attain working Energy models for simulation tools (BIM2BEPS) such as EnergyPlus and Modelica. For instance, Ramaji et al. \cite{Ramaji2020} proposed a BIM2BEPS transformation tool for conversion from the BIM-format International Foundation Classes (IFC) into an OpenStudio model.

Andriamamonjy et al. \cite{ANDRIAMAMONJY2018166} proposed an automated BIM2BEPS workflow between IFC and Modelica. The workflow applies an intermediate representation called Model View Definition (MVD), on which different checks are performed to ensure a given IFC file includes all required information. If the IFC passes this test, it can be directly translated into a working Modelica model. While the workflow is successfully applied to a test case study, the authors also highlighted potential barriers of adoption. Namely, that it relies greatly on information that is typically not available in status quo IFC files. 

Despite the significant efforts in BIM2BEPS research, many of the existing interoperability issues such as interdependence on extensive toolchains, inconsistent current practices by BIM practitioners, and loss of information still persist \cite{Andriamamonjy2018, CHONG20174114}. 

As a result of these shortcomings, data-driven modeling methods have recently gained increasing popularity for building modeling and simulation, due to their lower computational demands and a high potential for automation and integration in smart buildings where sensor networks and metered HVAC components collect large amounts of operational data. 

Data-driven modeling methods are typically divided into black-box and grey-box methods. Black-box models are purely data-driven, meaning that no domain-specific model structure is assumed. Given a set of data, specific algorithms are thus applied to find appropriate functional relationships that map system inputs to system outputs. Commonly used black-box models are Artificial Neural Networks (ANN) \cite{mtibaa2020a, fang2021a}, Support Vector Machines (SVM) \cite{b2005a}, Decision Trees \cite{yu2010a}, and state-space models \cite{ROYER201410850}. Due to the absence of domain-specific assumptions, these types of models have been successfully applied across many academic disciplines as well as in industry \cite{Shinde2018, ABIODUN2018e00938}. 

Typically, black-box models are considered lighter than white-box models in terms of computational complexity and simulation speed. For existing buildings with available historical data, black-box models have been used for aggregated heating and electricity load forecasting \cite{idowu2016a, guo2018a, pallonetto2022a}. These types of models are usually referred to as \textit{monolithic models}, i.e. whole-building models that do not incorporate any knowledge of the modeled building, besides historical weather and operational data such as heating or electricity consumption. While this offers convenient and fast implementation, there are also certain drawbacks to consider. If the building operation or design is changed, e.g. if system setpoints are changed or if the building is retrofitted, new data have to be collected and the model has to be re-trained. Furthermore, if the data available is scarce or of low quality, the performance of black-box models might be affected. A promising solution to this issue is the use of \textit{transfer learning}, a technique where knowledge is transferred from one domain to another \cite{PINTO2022100084}. Thereby, the knowledge contained in models that are trained on large amounts of data can be reused for training other models, given that the domains have a certain amount of overlap. This concept has been applied for both indoor temperature prediction \cite{Zhanhong2019} and energy prediction \cite{FANG2021119208}. 

One of the often highlighted drawbacks of black-box models is the lack of interpretability, i.e. the model structure and the obtained parameters have no direct physical meaning and the underlying cause of model behavior is thus unknown. This is opposed to grey-box models, the second category of data-driven models. As the name suggests, these types of models are a mix of white-box and black-box models where the model structure is based on physical principles, while the unknown parameters of this model structure are estimated through parameter identification techniques. The physics-derived model structure enables interpretation and validation of the obtained parameters from parameter estimation in a physical context, e.g. as demonstrated by Macarulla et al. \cite{Macarulla2017}. An example of the use of grey-box modeling in buildings is the thermal Resistor Capacity (RC) model, a thermal analog to electrical circuits used to model the thermal dynamics of buildings \cite{yu2019a, park2013a}. However, grey-box modeling is a general concept that can also be applied to individual building components such as heat pumps, recovery units, thermal storage tanks, cooling coils, etc., as demonstrated by Afram et al. \cite{AFRAM2015134}.


In general, the various recent studies on data-driven modeling of building systems and components using black-box and grey-box approaches report clear benefits regarding automation potential and flexibility as well as high prediction accuracy. Considering the discussed challenges of BIM2BEPS automation, these are key properties for applications in a dynamic DT-environment where close integration between data collected from IoT networks and simulation models is fundamental for delivering the outlined DT services. 




