%% 
%% Copyright 2007-2020 Elsevier Ltd
%% 
%% This file is part of the 'Elsarticle Bundle'.
%% ---------------------------------------------
%% 
%% It may be distributed under the conditions of the LaTeX Project Public
%% License, either version 1.2 of this license or (at your option) any
%% later version.  The latest version of this license is in
%%    http://www.latex-project.org/lppl.txt
%% and version 1.2 or later is part of all distributions of LaTeX
%% version 1999/12/01 or later.
%% 
%% The list of all files belonging to the 'Elsarticle Bundle' is
%% given in the file `manifest.txt'.
%% 

%% Template article for Elsevier's document class `elsarticle'
%% with numbered style bibliographic references
%% SP 2008/03/01
%%
%% 
%%
%% $Id: elsarticle-template-num.tex 190 2020-11-23 11:12:32Z rishi $
%%
%%
% \documentclass[preprint,12pt]{elsarticle}

%% Use the option review to obtain double line spacing
%% \documentclass[authoryear,preprint,review,12pt]{elsarticle}

%% Use the options 1p,twocolumn; 3p; 3p,twocolumn; 5p; or 5p,twocolumn
%% for a journal layout:
%% \documentclass[final,1p,times]{elsarticle}
% \documentclass[final,1p,times,twocolumn]{elsarticle}
%% \documentclass[final,3p,times]{elsarticle}
% \documentclass[final,3p,times,twocolumn]{elsarticle}
%% \documentclass[final,5p,times]{elsarticle}





\newcommand*\documentclassCustomStyleList{final,5p,times,twocolumn}
\newcommand\documentclassCustomCommand[1][]{\expandafter\documentclass\expandafter[#1]}
\documentclassCustomCommand[\documentclassCustomStyleList]{elsarticle}
\usepackage{temp_style}
\newcommand{\algorithmautorefname}{Algorithm}
\algrenewcommand\algorithmicindent{0.6em}%
\def\distanceToCaption{-7}

%%%%%%%%%%%%%%%%%%%%%%%%%%%%%%%%%%%%%%%%%%%%%%%%%%%%%%%%%%%%%%%%%%%%%%%%%
\makeatletter
% start with some helper code
% This is the vertical rule that is inserted
\newcommand*{\algrule}[1][\algorithmicindent]{%
  \makebox[#1][l]{%
    \hspace*{.2em}% <------------- This is where the rule starts from
    \vrule height .75\baselineskip depth .25\baselineskip
  }
}

\newcount\ALG@printindent@tempcnta
\def\ALG@printindent{%
    \ifnum \theALG@nested>0% is there anything to print
    \ifx\ALG@text\ALG@x@notext% is this an end group without any text?
    % do nothing
    \else
    \unskip
    % draw a rule for each indent level
    \ALG@printindent@tempcnta=1
    \loop
    \algrule[\csname ALG@ind@\the\ALG@printindent@tempcnta\endcsname]%
    \advance \ALG@printindent@tempcnta 1
    \ifnum \ALG@printindent@tempcnta<\numexpr\theALG@nested+1\relax
    \repeat
    \fi
    \fi
}
% the following line injects our new indent handling code in place of the default spacing
\patchcmd{\ALG@doentity}{\noindent\hskip\ALG@tlm}{\ALG@printindent}{}{\errmessage{failed to patch}}
\patchcmd{\ALG@doentity}{\item[]\nointerlineskip}{}{}{} % no spurious vertical space
% end vertical rule patch for algorithmicx
\makeatother
%%%%%%%%%%%%%%%%%%%%%%%%%%%%%%%%%%%%%%%%%%%%%%%%%%%%%%%%%%%%%%%%%%%%%%%%%


%% For including figures, graphicx.sty has been loaded in
%% elsarticle.cls. If you prefer to use the old commands
%% please give \usepackage{epsfig}

%% The amssymb package provides various useful mathematical symbols






%% The amsthm package provides extended theorem environments
%% \usepackage{amsthm}

%% The lineno packages adds line numbers. Start line numbering with
%% \begin{linenumbers}, end it with \end{linenumbers}. Or switch it on
%% for the whole article with \linenumbers.
%% \usepackage{lineno}

\journal{Energy and Buildings}

\begin{document}
% \algrenewcommand{\algorithmiccomment}[1]{\hskip3em$\rightarrow$ #1}


\linenumbers


\begin{frontmatter}

%% Title, authors and addresses

%% use the tnoteref command within \title for footnotes;
%% use the tnotetext command for theassociated footnote;
%% use the fnref command within \author or \address for footnotes;
%% use the fntext command for theassociated footnote;
%% use the corref command within \author for corresponding author footnotes;
%% use the cortext command for theassociated footnote;
%% use the ead command for the email address,
%% and the form \ead[url] for the home page:
%% \title{Title\tnoteref{label1}}
%% \tnotetext[label1]{}
%% \author{Name\corref{cor1}\fnref{label2}}
%% \ead{email address}
%% \ead[url]{home page}
%% \fntext[label2]{}
%% \cortext[cor1]{}
%% \affiliation{organization={},
%%             addressline={},
%%             city={},
%%             postcode={},
%%             state={},
%%             country={}}
%% \fntext[label3]{}

% \title{A Framework for Component-Based Energy Modeling of Building Digital Twins}
\title{An Ontology-based Innovative Energy Modeling Framework for Scalable and Adaptable Building Digital Twins}


%% use optional labels to link authors explicitly to addresses:
%% \author[label1,label2]{}
%% \affiliation[label1]{organization={},
%%             addressline={},
%%             city={},
%%             postcode={},
%%             state={},
%%             country={}}
%%
%% \affiliation[label2]{organization={},
%%             addressline={},
%%             city={},
%%             postcode={},
%%             state={},
%%             country={}}

\author[inst1]{Jakob Bjørnskov\corref{cor1}}
\ead{jabj@mmmi.sdu.dk}


\author[inst1]{Muhyiddine Jradi}



\cortext[cor1]{Corresponding author} 

\affiliation[inst1]{organization={SDU Center for Energy Informatics, Mærsk Mc-Kinney Møller Instituttet, University of Southern Denmark},%Department and Organization
            addressline={Campusvej 55}, 
            city={Odense M},
            postcode={5230}, 
            % state={State One},
            country={Denmark}}




\begin{abstract}
%% Text of abstract




Digitalization of buildings and the use of IoT sensing and metering devices are steadily increasing, offering new opportunities for more autonomous, efficient, and flexible buildings.  As part of this transformation and inspired by the added value demonstrated in other domains, the concept of a building digital twin that can monitor, simulate, manage, and optimize building operation has received increased interest. To aid such digital twin implementations, accurate and adaptable simulation models are required, which can effectively integrate and utilize the available data. However, traditional building modeling and digital practices, such as Building Information Modeling and white-box modeling tools, are not easily compatible with these requirements. This work presents an innovative and flexible energy modeling framework based on the SAREF ontology. With a basis in the SAREF4BLDG extension for buildings and the defined classes, different models are presented for a selection of typical systems and devices such as spaces, space heaters, dampers, coils, etc. Using the generic semantics and relations of the SAREF4SYST extension, a method for linking and simulating component models is then presented. A proof-of-concept of the modeling framework is provided, showing its application and feasibility to provide a dynamic simulation of the different systems and devices included in a demonstration case. Finally, a future line of work is identified considering the implementation of the modeling framework in an actual building case study, including integration with actual sensing equipment to demonstrate different digital twin services such as performance monitoring, strategy planning, and operational optimization. 





\end{abstract}

% %%Graphical abstract
% \begin{graphicalabstract}
% \includegraphics{grabs}
% \end{graphicalabstract}

% %%Research highlights
% \begin{highlights}
% \item Research highlight 1
% \item Research highlight 2
% \end{highlights}

\begin{keyword}
%% keywords here, in the form: keyword \sep keyword
Digital Twin \sep 
Data-driven \sep
Building Energy Model \sep 
Building Simulation \sep 
Ontology \sep
SAREF 


%% PACS codes here, in the form: \PACS code \sep code
% \PACS 0000 \sep 1111
%% MSC codes here, in the form: \MSC code \sep code
%% or \MSC[2008] code \sep code (2000 is the default)
% \MSC 0000 \sep 1111
\end{keyword}

\end{frontmatter}

%% \linenumbers

%% main text

\section{Introduction}
\label{sec:Introduction}

Driven by ambitious environmental goals and the promising potential for improved efficiency and automation by employing the Internet of Things (IoT) and Artificial Intelligence (AI) technologies, the building sector is currently undergoing fast-paced digitalization with increasing adoption of large sensing and metering networks. While the utilization of such data offers many opportunities, it also presents great challenges to status quo data management, building operation, and building energy modeling. As a consequence, the demand is steadily growing for more flexible and scalable data and energy models that integrate seamlessly. 

As part of the digitalization of buildings, Building Information Models (BIM) have for the past two decades played a major role, unlocking significant cost savings and improving efficiency by easing communication and information exchange between contractors in the design and construction phases of buildings \cite{ullah2019}. However, despite its demonstrated success, the BIM technology is now facing significant challenges as attempts are now made to extend it beyond its intended scope to fulfill the demands of IoT-integrated smart buildings. Boje et al. \cite{BOJE2020103179}, argued that BIM is not suitable for IoT integration due to the static legacy formats and standards used to represent data. Instead, they highlighted the emerging concept of Digital Twins (DT) in buildings as a promising solution for solving these interoperability issues. 


The DT concept was first used by the National Aeronautics and Space Administration (NASA), which defined it as a \textit{"comprehensive multi-physical, multi-scale, probabilistic simulation system for vehicles or systems"} \cite{Glaessgen2012}. Following, the DT concept has been adapted in various other scientific and engineering fields such as product manufacturing, medical sciences, and smart cities \cite{Guo2022}. Although DTs as a concept is starting to emerge in the building sector, the concept is still in its infancy with no clear consensus on a common definition and which services it should provide. 


% The work presented in this paper, is developed as part of the ongoing research project \textit{Twin4Build: A holistic Digital Twin platform for decision-making support over the whole building life cycle}. The project will in collaboration with industrial partners develop and implement a digital twin solution, 


\begin{figure*}[t!]
    \centering
    \includegraphics[width=1\linewidth, trim={0 0cm 0 2cm}, clip]{DT framework system diagram.png}
    \caption{The DT concept illustrated with typical systems and components present in buildings with their virtual counterpart and the bidirectional flow of data.}
    \label{fig:DT_concept}
\end{figure*}

In this work, we adopt the definition provided by Boje et al. \cite{BOJE2020103179} and Grieves \cite{Grieves2015}. That is, a DT refers to a concept with three main constituents; a physical system, a virtual system, and a flow of data linking these two systems. This is conceptualized in the context of a generic building energy system as shown in \autoref{fig:DT_concept}. Here, the physical system is the actual asset to be managed, i.e. a building including all systems, devices, spaces, occupants, etc. The virtual system employs relevant simulation models to emulate the behavior of the physical system as closely as possible. The physical and virtual systems are linked through a bidirectional data flow, where the virtual system receives raw data in the form of static design and topology information as well as dynamic sensor and meter data. The virtual system then processes this information to adapt, monitor, simulate, and optimize based on the employed simulation models. This processed information is then communicated back to the physical system for facility management. As a basis for the work presented in this paper and the envisioned DT, three fundamental services, targeting the operational phase of the building life cycle, have been defined as follows:




\begin{enumerate}[align=left]
    \item[\textbf{Service 1}] Efficient collection and management of data generated by sensors, meters, and IoT devices through a user-friendly open-standard context information model
    \item[\textbf{Service 2}] Automated continuous commissioning and performance monitoring in real-time to detect faults and malfunctions, ensuring a smarter and more cost-effective facility management
    \item[\textbf{Service 3}] Operational strategy planning support to enable more informed decision-making by running different control and management scenarios in a zero-risk virtual environment 
\end{enumerate}





In recent years, successful implementations of fault detection and performance monitoring, as listed in \textbf{Service 2}, have been reported numerous times with different approaches. Typically, fault detection methods are divided into two categories. The first category is model-based methods, which rely on comparisons between the simulated behavior of a system with actual measurements collected from the physical unit. The second category is model-free methods, which typically utilize only operational data collected from a system to find patterns that distinguish faulty from normal operation. Although model-free methods require no models to implement, their effectiveness is also often very limited in comparison to model-based methods when applied to large-scale distributed systems such as Heating Ventilation Air-Conditioning (HVAC) systems \cite{xiao2009a}. Therefore, most efforts in fault detection and continuous commissioning of buildings have considered model-based methods, often with the use of white-box simulation tools \cite{wang2013a, oneill2014a, markoska2016a, jradi2018a}. In addition, an accurate dynamic energy model enables simulation and optimization of future operational scenarios, as required by \textbf{Service 3}, aiding decision-makers with operating the building on a day-to-day basis, taking into account weather forecasts, expected occupancy, and price signals. This potentially increases indoor comfort, reduces operational costs, and improves the overall energy flexibility of the building. 


The added value of the outlined services is well-established and has been demonstrated numerous times in case-studies \cite{Metallidou2020,Bynum2008,GUNAY2019164}. However, using traditional energy modeling tools and being challenged by various interoperability issues, such implementations usually result in highly specialized solutions tailored to each specific building. This building-by-building approach, relies on manual workflows, is costly, hard to scale, and acts as a major barrier to broadly implementing such systems. 

Therefore, inspired by the added value provided in different domains, there is a need to promote and develop DT technology for the building sector to harness the benefits provided by advanced sensing and metering devices and provide the outlined services at scale. As a key milestone in such promotion, an automated and adaptable framework for energy modeling of buildings is needed to effectively integrate such models into a building DT. 


In this work, an innovative framework is proposed and presented for automated and adaptable energy model development to provide the simulation models required by building DTs. The framework builds upon the Smart Applications REFerence (SAREF) ontology \cite{saref} to ensure interoperability by using existing classes, concepts, and relations. A selection of SAREF classes is extended with adaptable data-driven grey-box and black-box models, to describe their dynamic behavior. Furthermore, a generic framework based on the SAREF4SYST extension is presented to describe how these component models interact based on simple topology information. To provide a proof-of-concept, a demonstration case is considered to illustrate the framework implementation considering the simple single-zone system shown in \autoref{fig:DT_concept}. 


The presented framework in this work is capable to serve as a basis for the development and implementation of scalable DTs for building applications, allowing the delivery of major services to enhance the energy efficiency and intelligence quotient of buildings. This includes performance monitoring, data management, performance optimization, and strategy planning.

As established, one of the core elements of DTs is accurate simulation models. Therefore, in the following section, an overview and discussion of different approaches for building energy modeling are provided. 

\section{Building Energy Modelling}

Over the years, considerable effort has been invested in developing large-scale energy simulation tools, e.g. EnergyPlus, DOE-2, and TRNSYS \cite{crawley2001a, winkelmann-a, klein2007a}. Models developed in such tools fall under the category of white-box models, which are based entirely on first-principles building physics, typically in the form of mass and energy balance equations, which often lead to large systems of Ordinary Differential Equations (ODE). However, being the case for the majority of large-scale white-box models, they require extensive information on the building and a substantial amount of resources and time to develop. Taking EnergyPlus as an example, a typical workflow consists of three time-consuming phases before an accurate energy model is obtained. First, the building geometry is defined with detailed geometry data from floor plans, cross-sections, 3D models, etc. Second, HVAC systems design data must be specified along with material and thermal properties of the envelope and occupancy, lighting, and equipment schedules of the spaces. Finally, to achieve acceptable prediction accuracy, the model must be calibrated using actual consumption data from the building, a process that often relies on a manual trial-and-error approach that requires domain-specific expertise from the modeler \cite{reddy2006a}. In addition to the high amount of resources and time that must be dedicated to the manufacturing of these high-resolution white-box models, high computational costs and execution time is also a well-known challenge, which makes them unsuitable for operational optimization purposes \cite{reynolds2018a, chen2022a}. 

Therefore, to reduce the amount of manual work of white-box model development, considerable efforts have been put into the automatic translation of the design and geometrical data available in BIM to attain working Energy models for simulation tools (BIM2BEPS) such as EnergyPlus and Modelica. For instance, Ramaji et al. \cite{Ramaji2020} proposed a BIM2BEPS transformation tool for conversion from the BIM-format International Foundation Classes (IFC) into an OpenStudio model.

Andriamamonjy et al. \cite{ANDRIAMAMONJY2018166} proposed an automated BIM2BEPS workflow between IFC and Modelica. The workflow applies an intermediate representation called Model View Definition (MVD), on which different checks are performed to ensure a given IFC file includes all required information. If the IFC passes this test, it can be directly translated into a working Modelica model. While the workflow is successfully applied to a test case study, the authors also highlighted potential barriers of adoption. Namely, that it relies greatly on information that is typically not available in status quo IFC files. 

Despite the significant efforts in BIM2BEPS research, many of the existing interoperability issues such as interdependence on extensive toolchains, inconsistent current practices by BIM practitioners, and loss of information still persist \cite{Andriamamonjy2018, CHONG20174114}. 

As a result of these shortcomings, data-driven modeling methods have recently gained increasing popularity for building modeling and simulation, due to their lower computational demands and a high potential for automation and integration in smart buildings where sensor networks and metered HVAC components collect large amounts of operational data. 

Data-driven modeling methods are typically divided into black-box and grey-box methods. Black-box models are purely data-driven, meaning that no domain-specific model structure is assumed. Given a set of data, specific algorithms are thus applied to find appropriate functional relationships that map system inputs to system outputs. Commonly used black-box models are Artificial Neural Networks (ANN) \cite{mtibaa2020a, fang2021a}, Support Vector Machines (SVM) \cite{b2005a}, Decision Trees \cite{yu2010a}, and state-space models \cite{ROYER201410850}. Due to the absence of domain-specific assumptions, these types of models have been successfully applied across many academic disciplines as well as in industry \cite{Shinde2018, ABIODUN2018e00938}. 

Typically, black-box models are considered lighter than white-box models in terms of computational complexity and simulation speed. For existing buildings with available historical data, black-box models have been used for aggregated heating and electricity load forecasting \cite{idowu2016a, guo2018a, pallonetto2022a}. These types of models are usually referred to as \textit{monolithic models}, i.e. whole-building models that do not incorporate any knowledge of the modeled building, besides historical weather and operational data such as heating or electricity consumption. While this offers convenient and fast implementation, there are also certain drawbacks to consider. If the building operation or design is changed, e.g. if system setpoints are changed or if the building is retrofitted, new data have to be collected and the model has to be re-trained. Furthermore, if the data available is scarce or of low quality, the performance of black-box models might be affected. A promising solution to this issue is the use of \textit{transfer learning}, a technique where knowledge is transferred from one domain to another \cite{PINTO2022100084}. Thereby, the knowledge contained in models that are trained on large amounts of data can be reused for training other models, given that the domains have a certain amount of overlap. This concept has been applied for both indoor temperature prediction \cite{Zhanhong2019} and energy prediction \cite{FANG2021119208}. 

One of the often highlighted drawbacks of black-box models is the lack of interpretability, i.e. the model structure and the obtained parameters have no direct physical meaning and the underlying cause of model behavior is thus unknown. This is opposed to grey-box models, the second category of data-driven models. As the name suggests, these types of models are a mix of white-box and black-box models where the model structure is based on physical principles, while the unknown parameters of this model structure are estimated through parameter identification techniques. The physics-derived model structure enables interpretation and validation of the obtained parameters from parameter estimation in a physical context, e.g. as demonstrated by Macarulla et al. \cite{Macarulla2017}. An example of the use of grey-box modeling in buildings is the thermal Resistor Capacity (RC) model, a thermal analog to electrical circuits used to model the thermal dynamics of buildings \cite{yu2019a, park2013a}. However, grey-box modeling is a general concept that can also be applied to individual building components such as heat pumps, recovery units, thermal storage tanks, cooling coils, etc., as demonstrated by Afram et al. \cite{AFRAM2015134}.


In general, the various recent studies on data-driven modeling of building systems and components using black-box and grey-box approaches report clear benefits regarding automation potential and flexibility as well as high prediction accuracy. Considering the discussed challenges of BIM2BEPS automation, these are key properties for applications in a dynamic DT-environment where close integration between data collected from IoT networks and simulation models is fundamental for delivering the outlined DT services. 





\section{Ontologies and metadata schemes for improved interoperability and building energy modeling}

Buildings have a high degree of heterogeneity and complexity, including a wide variety of systems and devices from different contractors and providers. This makes it difficult to develop applications and models that can be scaled and reused across different buildings and use-cases. This has caused the development of ontologies, which attempts to standardize and structure metadata and communication between devices to avoid the status-quo many-to-many translation between the multitude of protocols used. 

Pritoni et al. \cite{Pritoni2021} provided a review of 40 different ontology and metadata schemas for building design, energy modeling, and building operation applications. From these 40 different ontologies, five of the most popular ontologies were selected for a deeper comparison based on their applicability in three use cases; Energy auditing, automated fault detection and diagnosis, and optimal control of HVAC systems. The selected ontologies were SAREF, Semantic Sensor Network/Sensor, Observation, Sample, and Actuator (SSN/SOSA), Building Topology Ontology (BOT), Brick, and RealEstateCore (REC). 


Generally, it was highlighted that all five ontologies had varying degrees of missing concepts, which impacted their utility in the three use cases, i.e. choosing the right ontology depends on its intended application. For instance, BOT and SSN/SOSA missed central concepts for describing sensed properties, actuator idioms, and units of measurement. BOT, Brick, and REC missed equipment properties to describe e.g. nominal capacities, flowrates, efficiencies, etc. All ontologies except Brick missed schedule concepts for describing control strategies. Generally, Pritoni et al. conclude that none of the five ontologies accounts for all required concepts and that the ontologies must be customized and extended for certain applications. 



% BOT is designed to provide high-level topological relationships in buildings with concepts such as site, zone, space, and elements, and how these are related. However, 



% This work proposes a modeling framework using existing semantic definitions in the SAREF ontology and its extensions.

% previous attempts at aligning modelica and 
% For instance, Roxin et al. \cite{Roxin2021} investigated potential semantic alignment between the Modelica Standard Library (MSL) and the Smart Applications REFerence (SAREF) ontologies. However, the authors only found weak alignment between connections, classes, and properties due to fundamental differences in purpose and application. 



% efine both how different building components behave, but also how they interact in the context of energy modeling. 

% This is currently a missing link which is needed to provide the services outlined in \autoref{sec:Introduction}.


In this work, SAREF is chosen as a backbone for the energy modeling framework. SAREF was developed in 2015 by Daniele et al. \cite{Daniele2015} in close collaboration with industry to define a unified reference for recurring concepts and relations in the IoT domain. The core SAREF ontology revolves around the concept of devices, e.g. a lighting switch, sensor, or meter. The ontology has extensions spanning across multiple domains, e.g. SAREF4BLDG \cite{saref4bldg}, SAREF4CITY \cite{saref4city}, SAREF4SYST \cite{lefran2019a}, SAREF4ENER \cite{Daniele2020}, etc. Here, SAREF4BLDG is an extension dedicated to the building domain and is based on the ISO-published Industry Foundation Classes (IFC) standard \cite{unknown2018a}. The extension has 72 classes and 179 object properties that represent typical appliances and devices in buildings. Hence, this ontology allows an almost direct mapping between IFC and SAREF for most of these objects within the building domain. This is a great advantage compared with other relevant ontologies, which could potentially allow for significant reuse of existing well-structured information, contained in the frequently available IFC models for buildings. Another ontology that also has IFC mapping capabilities is IFCowl \cite{IFCowl}. However, as discussed by Rasmussen et al. \cite{Rasmussen2020}, IFCowl was designed to be backwards compatible with IFC and as a result inherits two main drawbacks; complexity and size. This makes the ontology both hard to understand and hard to use. In comparison, SAREF has a higher degree of modularity and its extension SAREF4BLDG is lighter in complexity and size. In addition, the multiple domains and broader scope of SAREF could allow for further expansion of the DT concept to consider larger systems such as districts or cities at a later stage. 




% \begin{figure*}[h!]
%     \centering
%     \includegraphics[width=1\linewidth, trim={0cm 0cm 9.4cm 0cm}, clip]{DT framework saref representation.png}
%     \caption{}
%     \label{fig:DT_saref}
% \end{figure*}

% \begin{figure*}[h]
%     \centering
%     \includegraphics[width=0.8\linewidth, trim={0cm 0cm 0cm 0cm}, clip]{SAREF4BLDG.png}
%     \caption{}
%     \label{fig:SAREF4BLDG}
% \end{figure*}

As is the case with the components in IFC, the component classes specified in SAREF4BLDG are data containers with no formal or mathematical description of their dynamic behavior or interaction with other components or systems. For instance, a \texttt{s4bldg:SpaceHeater} object holds the property \texttt{s4bldg:outputCapacity$^\textrm{op}$}, but it has no attached model which specifies how this property should be used to simulate actual heat consumption. 

Therefore, to build upon the SAREF4BLDG extension and ensure high flexibility, a modular and component-based modeling approach is proposed. Using the component definitions and class hierarchies from SAREF4BLDG, the energy model is thus split into a matching set of component models that can be automatically assembled and adapted, based on the specific use case. Each component should be able to learn from the collected data through various parameter estimation techniques and strive to improve prediction accuracy as the collected data accumulates. This approach will make it possible to continuously monitor and compare the performance of actual components in real-time with their digital counterparts in the digital twin environment, as required by \textbf{Service 2}. Furthermore, a highly modular and flexible model allows for the interchange of component models during the entire building lifecycle, which is necessary to maintain the Digital Twin as a close replica of the actual building, as the installed systems or building use might change over time. Therefore, the component models should be accurate enough to meet the defined services to a satisfactory level, but simple enough to allow for robust and automated model calibration. 











% ifcOWL is another ontology which aligns with the IFC standard. However, ifcOWL \cite{Rasmussen2020}

% Smart Energy Aware Systems (SEAS)

% SAREF 

% Flow Systems Ontology (FSO) \cite{KUKKONEN2022104067}






% \cite{Roxin2021} - modelica and SAREF semantics. Has good summary of digital twin and simulation.

















With a basis in the components depicted in \autoref{fig:DT_concept}, a description of the components and their properties from the SAREF4BLDG ontology is given in the Appendix \autoref{tab:s4bldg_components}. The superscripts \texttt{op} and \texttt{dp} stand for object property and data property, respectively. An object property points to another object, while a data property holds data in the form of standard types such as integer, string, floating-point, etc. A complete list of available components and a description of the listed properties can be found in the SAREF4BLDG documentation \cite{saref4bldg}. 


\section{Component modeling}
\label{sec:component_models}

When modeling buildings, it is common practice to take advantage of the major difference in physical time constants that are associated with the HVAC system and the building envelope and spaces. The HVAC system is often governed by fast-moving dynamics that reach equilibrium within seconds or minutes, while the large thermal inertia of the building envelope and spaces has very slow-moving dynamics. Therefore, it can often be justified to model the HVAC systems as quasi-steady-state systems, which decreases model complexity and lowers the computational burden considerably \cite{li2013a, jorissen2015a}. For these reasons, building simulation tools such as EnergyPlus is also heavily based on this modeling approach \cite{energy2021a}. Therefore, with a basis in the presented components in \autoref{tab:s4bldg_components}, potential candidate models have been found in the literature for implementation and detailed descriptions are included in \autoref{sec:appendix_A}. The key selection criteria for the models have been simplicity and applicability for data-driven calibration and parameter estimation. 





\begin{table}[h!]
\centering
\caption{Overview of inputs, outputs, parameters, and constants of potential mathematical models for the highlighted components.}
\label{tab:model_overview}
\resizebox*{\linewidth}{!}{%
    
    % Please add the following required packages to your document preamble:
% \usepackage{multirow}
\begin{tabular}{l|l|l|l|l}
\toprule
\textbf{Component}                        & \textbf{Inputs} & \textbf{Outputs} & \textbf{Parameters} & \textbf{Constants} \\ \midrule
\multirow{2}{*}{\textbf{Valve}}                    & $u_v \in [0,1]$ &   $\dot{m}_w \in \mathbb{R}^+$    & $N_v \in [0,1]$     &                    \\
                                          &                 &                  & $\dot{m}_{w,max} \in \mathbb{R}^+$   &                    \\ \midrule
\multirow{3}{*}{\textbf{Space heater}}             & $\dot{m}_w \in \mathbb{R}^+$&  $\dot{Q}_h \in \mathbb{R}^+$     & $U\!A \in \mathbb{R}^+$              & $c_{p,w}$          \\
                                          & $T_{w,in} \in \mathbb{R}$&                  &    $C_r \in \mathbb{R}^+$            &   $\Delta t$                 \\
                                          & $T_z \in \mathbb{R}$&                  &                     &                    \\ \midrule
\multirow{3}{*}{\textbf{Heating coil}}             &  $\dot{m}_a \in \mathbb{R}^+$    &  $\dot{Q}_{hc} \in \mathbb{R}^+$  &                     &    $c_{p,a}$                \\
                                          & $T_{a,set} \in \mathbb{R}$     &                  &                     &                    \\
                                          & $T_{a,in} \in \mathbb{R}$      &                  &                     &                    \\ \midrule
\multirow{3}{*}{\textbf{Cooling coil}}             &  $\dot{m}_a \in \mathbb{R}^+$    &  $\dot{Q}_{cc} \in \mathbb{R}^+$  &                     &    $c_{p,a}$                \\
                                          & $T_{a,set} \in \mathbb{R}$     &                  &                     &                    \\
                                          & $T_{a,in} \in \mathbb{R}$      &                  &                     &                    \\ \midrule
\multirow{5}{*}{\textbf{Air to air heat recovery}} &$\dot{m}_{a,sup} \in \mathbb{R}^+$&   $T_{a,sup,out} \in \mathbb{R}$& $\epsilon_{75\%,h} \in [0,1]$ &  $c_{p,a}$         \\
                                          &$\dot{m}_{a,exh} \in \mathbb{R}^+$&                  & $\epsilon_{75\%,c} \in [0,1]$ &                    \\
                                          &   $T_{a,sup,in} \in \mathbb{R}$&                  & $\epsilon_{75\%,h} \in [0,1]$ &                    \\
                                          &  $T_{a,exh,in} \in \mathbb{R}$ &                  & $\epsilon_{100\%,c} \in [0,1]$&                    \\
                                          &  $T_{a,set} \in \mathbb{R}$    &                  & $m_{a,max}$         &                    \\ \midrule
\multirow{7}{*}{\textbf{Fan}}                      &   $\dot{m}_a \in \mathbb{R}^+$   & $\dot{W}_{fan} \in \mathbb{R}^+$  & $c_1 \in \mathbb{R}$     &       $\rho_a$             \\
                                          &                 &                  & $c_2 \in \mathbb{R}$     &                    \\
                                          &                 &                  & $c_3 \in \mathbb{R}$     &                    \\
                                          &                 &                  & $c_4 \in \mathbb{R}$     &                    \\
                                          &                 &                  & $\dot{m}_{a,max}$   &                    \\
                                          &                 &                  & $\Delta P_{max}$    &                    \\
                                          &                 &                  & $\eta_{tot} \in [0,1]$&                    \\ \midrule
\multirow{3}{*}{\textbf{Damper}}                   & $u_d \in [0,1]$ &   $\dot{m}_a \in \mathbb{R}^+$    &       $a \in \mathbb{R}$           &                    \\
                                          &                 &                  &       $b \in \mathbb{R}$           &                    \\
                                          &                 &                  &       $c \in \mathbb{R}$           &                    \\ \midrule
\multirow{3}{*}{\textbf{Controller}}               &      $y_{set} \in \mathbb{R}$  &        $u \in [0,1]$       &      $K_p \in \mathbb{R}$          &                    \\ 
                                          &      $y_{meas} \in \mathbb{R}$        &                  &      $K_i \in \mathbb{R}$          &                    \\ 
                                          &                 &                  &      $K_d \in \mathbb{R}$          &                    \\ \midrule
\multirow{9}{*}{\textbf{Building Space}}           & $N_{occ} \in \mathbb{N}$       &        $C_z \in \mathbb{R}$     &      $K_{occ} \in \mathbb{R}^+$     &        $\Delta t$  \\ 
                                          &$\dot{m}_{a,sup} \in \mathbb{R}^+$&                  &      $m_z \in \mathbb{R}^+$          &                   \\ 
                                          &$\dot{m}_{a,exh} \in \mathbb{R}^+$&                  &                     &                    \\ 
                                          &                 &                  &                     &                    \\ 
                                          &     $T_o \in \mathbb{R}$       &      $T_z \in \mathbb{R}$       &         [-]         &                    \\ 
                                          &     $\Phi_s \in \mathbb{R}^+$    &                  &                          &                    \\ 
                                          &     $u_v \in [0,1]$       &                  &                          &                    \\ 
                                          &     $u_d \in [0,1]$       &                  &                          &                    \\ 
                                          &     $u_{s} \in [0,1]$    &                  &                          &                    \\ 
\bottomrule
\end{tabular}

}
\end{table}


To provide an overview of all the identified models, the inputs, outputs, parameters, and constants have been summarized for each model in \autoref{tab:model_overview}. The parameters for the models in \autoref{tab:model_overview} can either be given through design specifications or calculated through parameter estimation techniques if time-series data is available for the inputs and outputs. In the case where input-output data can be continuously provided, parameter estimation can be performed on a regular interval to attain a continuously improving performance on the different grey- and black-box component models. This method is demonstrated for indoor temperature forecasting by Ruano et al. \cite{RUANO2006682}.
Additionally, each physical component can be monitored closely, by comparing its operation with its digital replica to provide fault detection and continuous commissioning capabilities on a component level. If neither design specifications nor operational input-output data is available, default values must be assumed. 

% \textcolor{red}{
% As seen from \autoref{tab:model_overview}, many of the parameters are very model specific and are not available from the SAREF4BLDG ontology properties. As an example, consider the heat recovery model, which relies on the efficiency parameters $\epsilon_{75\%,h}$, $\epsilon_{100\%,h}$,$\epsilon_{75\%,c}$, $\epsilon_{100\%,c}$, or the fan model, which relies on the power coefficients $c_1$-$c_4$. However, this issue is anticipated in the SAREF4BLDG ontology, where it is proposed to add any missing functionality as part of a specialized subclass \cite{ETSI_s4bldg}.
% }



Based on the listed inputs and outputs in \autoref{tab:model_overview}, and the model descriptions in the previous sections, certain expected relationships between components arise. For instance, the space heater model in \autoref{tab:model_overview} expects the massflow $\dot{m}_w$, and the room temperature $T_{z}$ as input, which it can acquire from the valve model and the building space model, respectively. The valve model expects the valve position $u_v$ as input, which it can acquire from the controller component etc. In the following section, these types of relations are formalized using simple but broadly applicable concepts from the SAREF4SYST extension. 

%%%%%%%%%%%%%%%%%%%%%%%%%%%%%%%%%
% This  allows for the modeling of actual sensor and controller dynamics, e.g. by modeling a certain measurement frequency or the thermal capacitance of the sensor itself. However, in this work we assume that $y_{actual}=y_{measured}$. 


% It should be noted that the highlighted components and the presented models can easily be expanded or modified. 
\section{Modeling framework}
\label{sec:topology}

In the previous section, common building components were presented, along with generic and scalable component energy models. However, to express how the defined components interact, a domain-independent SAREF-compliant methodology for linking the inputs and outputs of component models is defined. The concept of having different self-contained component models defined by an input-output relationship and an overall framework that links these models share many similarities to the co-simulation framework defined in the Functional Mockup Interface (FMI) standard \cite{Blochwitz2011}. Here, models are exported from simulation tools as Functional Mockup Units (FMU) and are used as so-called slaves. The slaves are allowed to communicate at specific time intervals called communication points, managed by a master algorithm. Between communication points, the slaves are simulated independently. The modeling framework presented in this section shares many of the same principles as co-simulation and is thus well-aligned with the use of FMUs for modeling individual components.  

\subsection{Linking component models using SAREF4SYST}

SAREF has a dedicated extension called SAREF4SYST, which aims at providing a generic framework for representing the topology of systems and how they interact \cite{lefran2019a}. Through SAREF4SYST, the system topology can be expressed through three generic classes, \texttt{s4syst:System}, \texttt{s4syst:Connection}, and \texttt{s4syst:ConnectionPoint} along with nine applicable properties as shown in \autoref{fig:s4syst_overview}. 

\begin{figure}[h]
    \centering
    \includegraphics[width=1\linewidth, trim={0 0cm 0 0cm}, clip]{SAREF4SYST.png}
    \caption{Overview of the SAREF4SYST ontology extension with available classes and properties \cite{lefran2019a}.}
    \label{fig:s4syst_overview}
\end{figure}

As shown in \autoref{fig:s4syst_overview}, a \texttt{s4syst:System} object can be a subsystem of other \texttt{s4syst:System} objects specified through the property \texttt{s4syst:subSystemOf} or have subsystems itself specified through \texttt{s4syst:hasSubSystem}. Various forms of flows between systems, e.g. information or physical quantities can be expressed through the \texttt{s4syst:Connection} and \texttt{s4syst:ConnectionPoint} classes. In this work, the \texttt{s4syst:Connection} class is used to represent system outputs and the \texttt{s4syst:ConnectionPoint} class to represent system inputs. Hence, all inputs and outputs in \autoref{tab:model_overview} can be represented by \texttt{s4syst:ConnectionPoint} and \texttt{s4syst:Connection} objects respectively, while each component can be represented by \texttt{s4syst:System} objects. This concept is shown in \autoref{fig:data_structure}, where three SAREF4BLDG components are shown. As recommended in the SAREF4BLDG documentation \cite{ETSI_s4bldg}, each component is extended with its identified model as a specialized subclass from \autoref{sec:component_models}. Furthermore, these components all inherit from the \texttt{s4syst:System} superclass, enabling them to interact and communicate through the SAREF4SYST ontology. It should be noted that the full SAREF4BLDG class hierarchy is not included for clarity. 


\begin{figure}[!h]
    \centering
    
\newcommand*\documentclassCustomStyleList{final,5p,times,twocolumn}
\newcommand\documentclassCustomCommand[1][]{\expandafter\documentclass\expandafter[#1]}
\documentclassCustomCommand[class=elsarticle,\documentclassCustomStyleList]{standalone}
\usepackage{temp_style}



\begin{document}
    \resizebox{1\linewidth}{!}{%
        \begin{tikzpicture}    
            \draw (0, 2.5) node[inner sep=0] {
            \includegraphics[width=1\textwidth, trim={14.5cm 8.5cm 0cm 0cm}, clip]{DT_data_structure_legend.png}};
            
            \draw (0, 0) node[inner sep=0] {
            \includegraphics[width=1\textwidth, trim={14.5cm 8.5cm 0cm 0cm}, clip]{DT_data_structure.png}};
            
            \draw (1.4, 3.65) node {\large $T_z$};
            
            \draw (6.3, 0) node {\large $T_z$};
            
            \draw (6.3, -1.55) node {\large $\dot{m}_w$};
            \draw (1.3, -2.3) node {\large $\dot{m}_w$};
            
            
            % \draw (3.3, -1.8) node {\small \textbf{Test}};
            
            % \node[text=red,font=\fontsize{3}{3.5}\selectfont]{B};
        \end{tikzpicture}
    }
\end{document}
    \caption{Example of data structure showing how SAREF4SYST is used to link models contained in the SAREF4BLDG components through the \texttt{s4syst:System}, \texttt{s4syst:Connection}, and \texttt{s4syst:ConnectionPoint} classes.}
    \label{fig:data_structure}
\end{figure}



In \autoref{fig:data_structure}, simple information flows between the three components are shown. The \texttt{s4bldg:BuildingSpace} component points to a \texttt{s4syst:Connection} object through the \texttt{s4syst:connectedThrough} property. This \texttt{s4syst:Connection} object represents the indoor temperature output $T_z$ of the \texttt{s4bldg:BuildingSpace} model extension. Through the \texttt{s4syst:connectsSystemAt} property, it further points to a \texttt{s4syst:ConnectionPoint} object, which represents the input $T_z$ of the \texttt{s4bldg:SpaceHeater} model extension. This relation is expressed through the \texttt{s4syst:ConnectionPoint} property of the \texttt{s4syst:ConnectionPoint} object. The same concepts are then applied to relate the \texttt{s4bldg:Valve} water massflow output $\dot{m}_w$ to the \texttt{s4bldg:SpaceHeater} input $\dot{m}_w$. Altogether, this example shows how information is translated from model output to model input. Correspondingly, the inverse path from model input to model output is expressed through the \texttt{s4syst:connectsAt}, \texttt{s4syst:connectsSystemThrough}, and  \texttt{s4syst:connectsSystem} properties as shown in \autoref{fig:s4syst_overview}. This generic framework and the translation from output to input and vice versa at the \texttt{s4syst:Connection} and \texttt{s4syst:ConnectionPoint} objects provides the flexibility to exchange information between any type of models independently of model formulation and input-output terminology. Hence, the presented framework is not specific to the models shown in \autoref{tab:model_overview}, which could be modified or extended to cover other components such as \texttt{s4bldg:Chiller} or \texttt{s4bldg:Pump}, where models from external simulation tools and libraries could be considered, e.g. as FMUs.  

% As an example, the generic controller model extension from \autoref{fig:data_structure} could also be applied to control the damper position $u_d$ CO$_2$-concentration

% It is anticipated that for larger buildings, the system topology can potentially consist of hundreds or thousands of components and the dependency of inputs and outputs between components could vary significantly. 

\subsection{Model execution order}
\label{sec:simulation_framework}

The topology in \autoref{fig:data_structure} defines the required information exchange between components at a given timestep. This produces a scheduling problem, where it must be ensured that all required inputs are available when a model is executed during a simulation time step. For instance, in \autoref{fig:data_structure}, both the \texttt{s4bldg:BuildingSpace} and the \texttt{s4bldg:Valve} model must be executed before the \texttt{s4bldg:SpaceHeater} model can be executed. Such task scheduling problems are well-known in computer science and are formally known as topological sorting \cite{Pang2015}. Given a directed graph $G=(V,E)$, with $V$ being the set of nodes and $E$ being the set of edges, topological sorting is the ordering of nodes such that for every edge $(u,v) \in E$ the node $u$ comes before the node $v$. This is only possible if the graph has no cycles, i.e. if it is a Directed Acyclic Graph (DAG), in which case it is guaranteed to have at least one solution \cite{Pang2015,Renkun2014}. Hence, if the \texttt{s4syst:System} instances are interpreted as nodes, and the \texttt{s4syst:Connection}, and \texttt{s4syst:ConnectionPoint} pairs are interpreted as directed edges, the correct execution order can be found using topological sorting, given that the graph is acyclic. 

\subsection{Cycle removal procedure}
The no-cycles condition of topological sorting is essential to consider when applying the presented methodology to model actual buildings as feedback control loops are an integral part of all building automation systems. When such control loops are modeled, cycles are introduced. An example of a cycle produced by a controller is shown in \autoref{fig:data_structure_cycle}. Here, a \texttt{s4bldg:BuildingSpace} model predicts CO$_2$-concentration $C_z$, which is used as input $y$ in a \texttt{saref:Sensor} model. The output $y_{meas}$ is then used as input in a \texttt{s4bldg:Controller} model. An appropriate input signal $u$ is then computed, which is sent to a \texttt{s4bldg:Damper} model to calculate the resulting airflow $m_a$. Finally, this value is sent back to the \texttt{s4bldg:BuildingSpace} model for recalculation of indoor CO$_2$-concentration $C_z$ at the next time step, completing the cycle. To systematically deal with such cycles, an algorithm is proposed and is presented in \autoref{alg:remove_cycles}. Given the set of all \texttt{s4syst:System} objects $H$ and an initially identical set $H^*$, the algorithm iterates through the set of all \texttt{s4bldg:Controller} components $U \in H^*$, identifies reachable components $V_u$ from $u \in U$, and removes all edges from $v \in V_u$ to the controlled component $u$\texttt{.controlsProperty.isPropertyOf} if such edges exist. For instance, applying \autoref{alg:remove_cycles} on the simple system shown in \autoref{fig:data_structure_cycle} the \texttt{Supply damper} would be identified as a reachable component from the \texttt{CO2 controller}. The \texttt{Supply damper} is connected with the controlled component \texttt{Space}, and this connection is thus removed.


\begin{figure}[H]
    \centering
    
\newcommand*\documentclassCustomStyleList{final,5p,times,twocolumn}
\newcommand\documentclassCustomCommand[1][]{\expandafter\documentclass\expandafter[#1]}
\documentclassCustomCommand[class=elsarticle,\documentclassCustomStyleList]{standalone}
\usepackage{temp_style}



\begin{document}
    \resizebox{1\linewidth}{!}{%
        \begin{tikzpicture}    
            \draw (0, 2.2) node[inner sep=0] {
            \includegraphics[width=1\textwidth, trim={14.5cm 5.5cm 0cm 0.3cm}, clip]{DT_data_structure_legend.png}};
            
            \draw (0, 0) node[inner sep=0] {
            \includegraphics[width=1\textwidth, trim={14.5cm 5.7cm 0cm 0.1cm}, clip]{DT_data_structure_cycle.png}};

            \draw (6.3, 5.1) node {\large $\dot{m}_{a,sup}$};
            \draw (1.4, 4.8) node {\large $C_z$};

            \draw (6.3, 1.6) node {\large $y$};
            \draw (1.3, 1.3) node {\large $y_{meas}$};
            
            \draw (6.3, -1) node {\large $y_{meas}$};
            \draw (1.3, -1.4) node {\large $u$};
            
            \draw (6.3, -3.6) node {\large $u_d$};
            \draw (1.3, -3.9) node {\large $\dot{m}_a$};
            
            
            
            % \draw (3.3, -1.8) node {\small \textbf{Test}};
            
            % \node[text=red,font=\fontsize{3}{3.5}\selectfont]{B};
        \end{tikzpicture}
    }
\end{document}
    \caption{Example of a cycle produced by a controller, which actuates the opening position of a supply airflow damper of a space. Some model inputs are neglected for clarity (e.g. the setpoint for the controller).}
    \label{fig:data_structure_cycle}
\end{figure}



\begin{algorithm}[H]
\caption{Removes cycles created by feedback control loops. DFS($V$, $s$) is a recursive depth-first search to find the set of reachable components $V$ from the component $s$.}
\label{alg:remove_cycles}

\begin{algorithmic}[1]
    \State Let $H$ be the set of all instances with superclass \texttt{s4syst:System}
    \State Let $H^*$ contain a copy of all the elements in $H$ such that the elements in $H^*$ can be modified to contain no cycles
    \State Let $U$ be the subset of all instances with the class \texttt{s4bldg:Controller} of $H^*$ 
    % \State Let $w_u$ be the component controlled by the controller $u \in U$ (i.e. the component supplying the $y$ input to the controller, cf. \autoref{tab:model_overview})
    % \newline
    
    \Function{DFS}{$V$, $s$}
        \State $V \gets V \cup \{s\}$
        
        \ForAll{$c \in s$.connectedThrough}
            \State $cp \gets c$.connectsSystemAt
            \State $s^* \gets cp$.connectionPointOf 
            \If{$s^* \not\in V$}
                \State $V \gets$ DFS($V$, $s^*$)
            \EndIf
        \EndFor
        
        \State \textbf{return} $V$
    \EndFunction
    % \newline

    \ForAll{$u \in U$}
        \State $w_u \gets u$.controlsProperty.isPropertyOf
        \State $V_u \gets$ DFS($\emptyset$, $u$)
        \ForAll{$s \in V_u$}
            \ForAll{$c \in$ $s$.connectedThrough}
                \State $cp \gets c$.connectsSystemAt
                \State $s^* \gets cp$.connectionPointOf 
                \If{$w_u = s^*$}
                    \State $w_u$.connectsAt $\gets w_u$.connectsAt $\setminus \; \{cp\}$ 
                    \State $s$.connectedThrough $\gets s$.connectedThrough $\setminus \; \{c\}$
                \EndIf
            \EndFor

        \EndFor
    \EndFor
\end{algorithmic}
\end{algorithm}

 



\subsection{Topological sorting procedure}
 

\begin{algorithm}[h!]
\caption{A SAREF4SYST-specific adaptation of Kahn's topological sorting algorithm \cite{Kahn1962TopologicalSO}.}
\label{alg:topological_sorting}

\begin{algorithmic}[1]
    
    
    \State Let $f : H^* \to H$ be a function that maps from the modified components $h^* \in H^*$ to the original components $h \in H$
    \State Let $S$ be the subset of $H^*$ with no inputs, i.e. \newline$S = \{s \in H^* : |s$.connectsAt$| = 0\}$
    \State Let $L\gets ()$ be an empty sequence which will be populated with the components of $H$ in a sorted order
    \State Let $P\gets()$ be an empty sequence which will be populated with the priority of the components in the sequence $L$
    \State Let $p \gets 0$ be the priority counter
    \State The notation $X^\frown(x)$ is used to append the element $x$ to the sequence $X$
    % \newline
    
    % \State Let $L$ be an empty sequence, which will be populated with the elements of $H$ in a sorted order.
    % An element $s$ is added to the sequence $L$ using the notation $L^\frown (s)$.
    
    \While{$|S|>0$}
        \State $S^* \gets \emptyset$
        \ForAll{$s \in S$}
            % \State $L \gets [L, s]$
            \State $L \gets L^\frown(\;f(s)\;)$
            \State $P \gets P^\frown(p)$
            \ForAll{$c \in $ $s$.connectedThrough}
                \ForAll{$cp \in $ $c$.connectsSystemAt}
                    \State $s^*$ $\gets$ $cp$.connectionPointOf
                    
                    \State $s^*$.connectsAt $\gets s^*$.connectsAt $\setminus \; \{cp\}$ 
                    
    
                    \If{$|s^*$.connectsAt$| = 0$}
                        \State $S^* \gets S^* \cup \{s^*\}$
                    \EndIf
                
                \EndFor
                

            \EndFor
        \EndFor
        \State $p \gets p + 1$
        \State $S \gets S^*$
    \EndWhile
\end{algorithmic}
\end{algorithm}





After applying \autoref{alg:remove_cycles}, topological sorting can be applied on the modified set of components $H^*$. This is shown in \autoref{alg:topological_sorting} where an adaptation of Kahn's topological sorting algorithm is presented, which matches the semantics of SAREF4SYST. The input of this algorithm is the set $H^*$, while the output is a sequence $L$ with all the components in $H$ in a topologically sorted order. The algorithm recursively adds all mapped components $f(s) \in H$ with $|s$.connectsAt$| = 0$ (i.e. the component $s$ has no inputs) to a sequence $L$ while removing all edges originating from $s$. In addition, a sequence $P$ is created which holds integers starting from 0 that represent the execution priority of the components in $L$. Components in $L$ with the same priority can be executed in parallel. For instance, applying \autoref{alg:topological_sorting} on the simple system shown in \autoref{fig:data_structure} produces the sequence $L=(\texttt{Valve}, \;\texttt{Space}, \;\texttt{Space \!heater})$ and the priority sequence $P=(0,0,1)$. This implies that the \texttt{Valve} and \texttt{Space} models can be executed in parallel and that they must be executed before the \texttt{Space \!heater} model. 

\subsection{Simulation procedure}

When the execution sequence $L$ is found using \autoref{alg:topological_sorting}, the system can be simulated by simply traversing through this sequence of components, while gathering the needed inputs and executing the attached model. Executing this routine once completes one system timestep of length $\Delta t$ and can be repeated indefinitely given the required inputs, such as weather data and schedule values, are available. This is formulated in \autoref{alg:simulation}, where each component model is assumed to have an input and output dictionary which are indexed by the properties \texttt{inputName} and \texttt{outputName} of the associated \texttt{s4syst:ConnectionPoint} and \texttt{s4syst:Connection} objects, respectively. These property names are simply the mathematical notation for the inputs and outputs shown in \autoref{tab:model_overview} translated into a representative name, e.g. "indoorTemperature" to represent $T_z$. In addition, each component model is assumed to have a \texttt{do\_step()} method, implementing the governing equations defined in \autoref{sec:component_models} or executing external models or libraries such as exported FMU models. 

\begin{algorithm}[h]
\caption{An algorithm for simulating building devices and systems given the execution sequence $L$.}
\label{alg:simulation}

\begin{algorithmic}[1]
    \State Let $t_{start}$ and $t_{end}$ be the start and end simulation time
    \State Let $\Delta t$ be the simulation time step
    \State $t \gets t_{start}$
    
    \While{$t<t_{end}$}
        \For{$i\gets 1 \; ... \; |L|$}
        \State $s \gets L_i$
            \ForAll{$c\!p \in s$.connectsAt}
                \State $c \gets c\!p$.connectsSystemThrough
                \State $s^*$ $\gets c$.connectsSystem
                \State $s$.input[$cp$.inputName] $\gets s^*$.output[$c$.outputName]
                % connection = connection_point.connectsSystemThrough
                % connected_component = connection.connectsSystem
                % component.input[connection_point.recieverPropertyName] = connected_component.output[connection.senderPropertyName]
                
            \EndFor
            \State $s$.output $ \gets s$.do\_step$(s$.input$)$
        \EndFor
        \State $t \gets t + \Delta t$
    \EndWhile
\end{algorithmic}
\end{algorithm}

% Each component is furthermore assumed to have 


% Note that this workflow has been validated on much larger systems on the order of 100 \texttt{s4bldg:BuildingSpace} instances. 

% Note that all components $v \in V_u$ where edges are removed in \autoref{alg:remove_cycles} must have specified an initial value for the properties represented by these edges as these edges still exist in the actual system . 

% It is common to use pseudorandom binary sequences (PRBS) to obtain training data for grey-box models \textcolor{red}{REF}. This model is trained on historical data collected fro normal building operation. 


% \begin{figure}[h!]
%     \centering
%     \includegraphics[width=1\linewidth, trim={0 0cm 0 6cm}, clip]{BEM framework overall.png}
%     \caption{}
%     \label{fig:BEM_overview}
% \end{figure}




% \begin{figure*}[h]
%     \centering
%     \includegraphics[width=1\linewidth, trim={0cm 0cm 0cm 0cm}, clip]{system_graph.png}
%     \caption{}
%     \label{fig:}
% \end{figure*}














\section{Demonstration case and implementation}



% \begin{figure*}[h!]
%     \centering
%     \includegraphics[width=1\linewidth, trim={0 0cm 0 0}, clip]{system_graph_1space_1v_1h_0c.png}
%     \caption{System graph for the demonstration example. The \texttt{s4syst:Connection} and \texttt{s4syst:ConnectionPoint} objects have been condensed into one edge for clarity.}
%     \label{fig:system_graph}
% \end{figure*}


% \begin{figure*}
%      \centering
%      \begin{subfigure}{1\linewidth}
%          \centering
%         %  \includegraphics[width=\textwidth]{system_graph_1space_1v_1h_0c.png}
%         
\newcommand*\documentclassCustomStyleList{final,5p,times,twocolumn}
\newcommand\documentclassCustomCommand[1][]{\expandafter\documentclass\expandafter[#1]}
\documentclassCustomCommand[class=elsarticle,\documentclassCustomStyleList]{standalone}
\usepackage{temp_style}



\begin{document}
    \resizebox{1\linewidth}{!}{%
        \begin{tikzpicture}    
            \draw (0, 0) node[inner sep=0] {
            \includegraphics[width=1\textwidth, trim={0 0cm 0cm 0cm}, clip]{system_graph_1space_1v_1h_0c.png}};
            
            \draw (13, 0) node[inner sep=0] {
            \includegraphics[width=0.2\textwidth, trim={69cm 0cm 69cm 0cm}, clip]{execution_graph_1space_1v_1h_0c.png}};
            
            \draw[-Triangle, very thick](9.5, 0) -- (10.5, 0);
            
            % \draw (1.4, 5.5) node {\large $T_z$};
            
            % \draw (6.3, 1.85) node {\large $y$};
            % \draw (1.3, 1.5) node {\large $u$};
            
            % \draw (6.3, -2.3) node {\large $u_v$};
            % \draw (1.3, -2.6) node {\large $\dot{m}_w$};
            
            % \draw (6.3, -5.1) node {\large $\dot{m}_w$};

        \end{tikzpicture}
    }
\end{document}
%          \caption{}
%          \label{fig:}
%      \end{subfigure}
     
%      \begin{subfigure}{1\linewidth}
%          \centering
%         %  \includegraphics[width=\textwidth]{system_graph_10space_1v_1h_0c.png}
%         
\newcommand*\documentclassCustomStyleList{final,5p,times,twocolumn}
\newcommand\documentclassCustomCommand[1][]{\expandafter\documentclass\expandafter[#1]}
\documentclassCustomCommand[class=elsarticle,\documentclassCustomStyleList]{standalone}
\usepackage{temp_style}



\begin{document}
    \resizebox{1\linewidth}{!}{%
        \begin{tikzpicture}    
            \draw (0, 0) node[inner sep=0] {
            \includegraphics[width=1.15\textwidth, trim={0 0cm 0cm 0cm}, clip]{system_graph_10space_1v_1h_0c.png}};
            
            \draw (13, 0) node[inner sep=0] {
            \includegraphics[width=0.09\textwidth, trim={84cm 0cm 84cm 0cm}, clip]{execution_graph_10space_1v_1h_0c.png}};
            
            \draw[-Triangle, very thick](11, 0) -- (12, 0);
            
            % \draw (1.4, 5.5) node {\large $T_z$};
            
            % \draw (6.3, 1.85) node {\large $y$};
            % \draw (1.3, 1.5) node {\large $u$};
            
            % \draw (6.3, -2.3) node {\large $u_v$};
            % \draw (1.3, -2.6) node {\large $\dot{m}_w$};
            
            % \draw (6.3, -5.1) node {\large $\dot{m}_w$};

        \end{tikzpicture}
    }
\end{document}
%          \caption{}
%          \label{fig:}
%      \end{subfigure}
     
     
     
%     %  \begin{subfigure}{1\linewidth}
%     %      \centering
%     %      \includegraphics[width=\textwidth]{system_graph_21space_2v_1h_0c.png}
%     %      \caption{}
%     %      \label{fig:}
%     %  \end{subfigure}
     
     
%         \caption{3 examples on the use of }
%         \label{fig:3_system_topology_examples}
% \end{figure*}


% One of the case studies on which the proposed methodology will be tested is a 8500 m$^2$ university building located in Denmark. The building was commissioned in 2015 and is highly energy efficient and well-instrumented with sensors. 
To validate the proposed modeling framework, the single-zone system shown in \autoref{fig:DT_concept} is considered as a demonstration case. 
The modeling framework is implemented in Python as a reusable library \cite{EMF}, following the semantic structure and guidelines provided by the SAREF ontology. However, to accurately model the system, basic topological information is required. To ensure that the model can easily be constructed and adapted without manually specifying all the logical connections between models as presented in \autoref{sec:topology}, a simple model-independent representation is instead used as a starting point. Based on this representation, the logical connections between models can then be inferred based on simple rules. For this purpose, the single-zone system has been formulated using the SAREF ontology semantics and concepts as shown in \autoref{fig:DT_saref} only considering the actual topology of the physical system.

\begin{figure*}[t!]
    \centering
    
\newcommand*\documentclassCustomStyleList{final,5p,times,twocolumn}
\newcommand\documentclassCustomCommand[1][]{\expandafter\documentclass\expandafter[#1]}
\documentclassCustomCommand[class=elsarticle,\documentclassCustomStyleList]{standalone}
\usepackage{temp_style}



\begin{document}
    \resizebox{1\linewidth}{!}{%
        \begin{tikzpicture}    
            \draw (0, 4.8) node[inner sep=0] {
            \includegraphics[width=0.5\textwidth, trim={14.5cm 8.5cm 0cm 0cm}, clip]{DT_data_structure_legend.png}};
            
            \draw (0, 0) node[inner sep=0] {
            \includegraphics[width=1\textwidth, trim={0cm 0cm 4cm 0cm}, clip]{DT framework saref representation.png}};
        
            % \draw (3.3, -1.8) node {\small \textbf{Test}};
            
            % \node[text=red,font=\fontsize{3}{3.5}\selectfont]{B};
        \end{tikzpicture}
    }
\end{document}
    \caption{SAREF representation of the demonstration case, including the required topology information for the involved components. The representation incorporates concepts from both SAREF core, SAREF4BLDG, and SAREF4SYST.}
    \label{fig:DT_saref}
\end{figure*}

\subsection{Available topology information}

The system consists of one \texttt{s4bldg:BuildingSpace} instance, containing one \texttt{s4bldg:SpaceHeater}, one \texttt{s4bldg:ShadingDevice}, one \texttt{s4bldg:Valve}, two \texttt{s4bldg:Damper}, two \texttt{s4bldg:Controller}, and two \texttt{saref:Sensor} instances as represented by the \texttt{s4bldg:contains} property. The \texttt{s4bldg:BuildingSpace} instance is linked with one \texttt{saref:Temperature} instance representing indoor temperature and one \texttt{saref:CO2} instance representing CO2-concentration through \texttt{saref:hasProperty}, which is inherited through the \texttt{saref:FeatureOfInterest} super class. The \texttt{saref:CO2} subclass of \texttt{saref:Property} has been added as an extension to the core SAREF ontology similar to the work carried out by Weerdt et al. \cite{Weerdt2021}, as it is not included by default. The measured properties of the two \texttt{saref:Sensor} instances are then determined through the \texttt{saref:measuresProperty}, pointing either to the \texttt{saref:Temperature} instance or the \texttt{saref:CO2} instance. Similarly, fashion, the controlled properties of the \texttt{s4bldg:Controller} instances are determined through the \texttt{saref:controlsProperty}. To determine which devices are actuated by the controllers, each \texttt{s4bldg:Controller} instance is associated with a \texttt{saref:LevelControlFunction} instance (subclass of \texttt{saref:Function}) through the \texttt{saref:hasFunction} property, which further points to a \texttt{saref:SetLevelCommand} instance (subclass of \texttt{saref:Command}) through the \texttt{saref:hasCommand} property. Finally, the \texttt{saref:SetLevelCommand} instances each point to a \texttt{saref:MultiLevelState} instance (subclass of \texttt{saref:State}) through the \texttt{actsUpon} property. The specific subclasses used here indicate that the actuated \texttt{s4bldg:Damper} and \texttt{s4bldg:Valve} instances can be regulated with a level-setting, e.g. from 0-100\%. Alternatively, for representing binary control of other objects such as windows or doors, the \texttt{saref:OpenCloseFunction}, \texttt{saref:OpenCommand}, \texttt{saref:CloseCommand}, \texttt{saref:OpenState}, and \texttt{saref:CloseState} could be used instead. For the shown system, the supply and exhaust dampers share the same state, resulting in balanced supply and exhaust airflows. However, for controlling supply and exhaust airflows separately, each damper instance should have its own \texttt{saref:MultiLevelState} instance each associated with two different \texttt{s4bldg:Controller} instances. 

\begin{figure*}[t!]
    \centering
    
\newcommand*\documentclassCustomStyleList{final,5p,times,twocolumn}
\newcommand\documentclassCustomCommand[1][]{\expandafter\documentclass\expandafter[#1]}
\documentclassCustomCommand[class=elsarticle,\documentclassCustomStyleList]{standalone}
\usepackage{temp_style}



\begin{document}
    \resizebox{1\linewidth}{!}{%
        \begin{tikzpicture}

            
        

            \draw (0, 0) node[inner sep=0] {
            \includegraphics[width=1.3\textwidth, trim={0 0cm 0cm 5cm}, clip]{system_graph_1space_1v_1h_0c_w_area.png}};
            
            
            \draw (-2.4, -14) node[inner sep=0] {
            \includegraphics[width=1\textwidth, trim={0 0cm 0cm 0cm}, clip]{system_graph_1space_1v_1h_0c_no_cycles_w_area.png}};
            
            % \draw (10.5, -11.5) node[inner sep=0] {
            % \includegraphics[width=0.38\textwidth, trim={0cm 0cm 0cm 0cm}, clip]{execution_graph_1space_1v_1h_0c_w_area.png}};


            
            
            
            % \draw[-latex,color=black, line width=0.7mm](-1,-4.3) -- (-1,-6);

            \draw (-0.7, -7) node {\large \makecell[l]{\textbf{1. Apply Algorithm 1 to remove cycles produced by feedback control loops}\\ \textbf{to obtain the modified set of components $H^*$ suitable for topological sorting}}};
            
            \draw[->,color=black, line width=0.5mm](-8,-6.4)
            arc
            [
                start angle=135,
                end angle=225,
                x radius=1.9cm,
                y radius =2.5cm
            ];
            
            
            
            
            
            \draw (11, -20.5) node {\large \makecell[l]{\textbf{2. Apply Algorithm 2 on \boldmath$H^*$}\\ \textbf{to obtain the sequence $L$}}};
            
            \draw[->,black, line width=0.5mm] (5.9,-19.9) arc
            [
                start angle=230,
                end angle=360,
                x radius=2cm,
                y radius =1cm
            ];
            
            
            \draw (9.8, 3.5) node {\large \makecell[l]{\textbf{3. Apply the obtained sequence $L$ to simulate}\\ \textbf{the original system using Algorithm 3}}};
            
            \draw[->,black, line width=0.5mm] (12,-4) arc
            [
                start angle=0,
                end angle=100,
                x radius=3cm,
                y radius =6cm
            ];
            
            % \draw[-Triangle, very thick](0, -3) -- (0, -5);
            
            % \draw (1.4, 5.5) node {\large $T_z$};
            \draw (-1, 3.8) node {\Huge $H$};
            \draw (-1.5, -8.9) node {\Huge $H^*$};
            \draw (0, -10.5) node {\huge $S$};
            % \draw (10.5, -5) node {\Huge $L$};
            % \draw (1.3, 1.5) node {\large $u$};
            




            \draw (10.4, -11.5) node[inner sep=0,opacity=0,rounded corners=2.9cm,save path=\mypath]
             {\includegraphics[width=0.4\textwidth, trim={3cm -0.3cm 2.1cm 0cm}, clip]{execution_graph_1space_1v_1h_0c_w_area.png}};
            \clip[use path=\mypath]; 
            \draw (10.4, -11.5) node[inner sep=0pt,opacity=1]
            {\includegraphics[width=0.4\textwidth, trim={3cm -0.3cm 2.1cm 0cm}, clip]{execution_graph_1space_1v_1h_0c_w_area.png}};
            \draw (10.4, -5.4) node {\Huge $L$};

        \end{tikzpicture}
    }
\end{document}



% \begin{document}
%     \resizebox{1\linewidth}{!}{%
%         \begin{tikzpicture}    
        
        
%             \draw (0, 0) node[inner sep=0] {
%             \includegraphics[width=1\textwidth, trim={0 0cm 0cm 3.8cm}, clip]{system_graph_1space_1v_1h_0c_w_area_rect.png}};
            
            
%             \draw (0, -11) node[inner sep=0] {
%             \includegraphics[width=0.93\textwidth, trim={0 0cm 0cm 2cm}, clip]{system_graph_1space_1v_1h_0c_no_cycles_w_area_rect.png}};
            
%             \draw (13, -7) node[inner sep=0] {
%             \includegraphics[width=0.4\textwidth, trim={2.3cm 0cm 2.2cm 0cm}, clip]{execution_graph_1space_1v_1h_0c_w_area_rect.png}};
            
            
%             % \draw[-latex,color=black, line width=0.7mm](-1,-4.3) -- (-1,-6);

%             \draw (-0.6, -5.2) node {\large \makecell[l]{\textbf{Apply \autoref{alg:remove_cycles} to remove cycles produced by feedback control loops}\\ \textbf{to obtain DAG suitable for topological sorting}}};
            
%             \draw[->,color=black, line width=0.5mm](-7,-4.2)
%             arc
%             [
%                 start angle=120,
%                 end angle=240,
%                 x radius=0.9cm,
%                 y radius =1.1cm
%             ];
            
            
            
            
            
%             \draw (14, -15.5) node {\large \makecell[l]{\textbf{Apply \autoref{alg:topological_sorting} on \boldmath$H^*$}\\ \textbf{to obtain the sequence $L$}}};
            
%             \draw[->,black, line width=0.5mm] (8.2,-15.3) arc
%             [
%                 start angle=230,
%                 end angle=360,
%                 x radius=2cm,
%                 y radius =1.5cm
%             ];
            
            
%             \draw (13.7, 3.3) node {\large \makecell[l]{\textbf{Apply the obtained sequence $L$ to simulate}\\ \textbf{the original system using \autoref{alg:simulation}}}};
            
%             \draw[->,black, line width=0.5mm] (12.8,0.2) arc
%             [
%                 start angle=10,
%                 end angle=90,
%                 x radius=3.3cm,
%                 y radius =2.5cm
%             ];
            
%             % \draw[-Triangle, very thick](0, -3) -- (0, -5);
            
%             % \draw (1.4, 5.5) node {\large $T_z$};
%             \draw (0, 4.4) node {\Huge $H$};
%             \draw (0, -6.9) node {\Huge $H^*$};
%             \draw (-1, -8.3) node {\huge $S$};
%             \draw (13, -1) node {\Huge $L$};
%             % \draw (1.3, 1.5) node {\large $u$};
            
%             % \draw (6.3, -2.3) node {\large $u_v$};
%             % \draw (1.3, -2.6) node {\large $\dot{m}_w$};
            
%             % \draw (6.3, -5.1) node {\large $\dot{m}_w$};

%         \end{tikzpicture}
%     }
% \end{document}
    \caption{Overview of \autoref{alg:remove_cycles}, \autoref{alg:topological_sorting}, and \autoref{alg:simulation} applied to the demonstration case.}
    \label{fig:Topology_1space_1v_1h_0c}
\end{figure*}

To describe the topology of the ventilation system components and their order of placement in the flow path, the \texttt{s4syst:connectedTo} property is used, which is inherited from the \texttt{s4syst:System} class. This property is symmetric, meaning that two connected components both point to each other with the \texttt{connectedTo} property (although only one path is shown in \autoref{fig:DT_saref}). Hence, the \texttt{s4bldg:AirToAirHeatRecovery} instance is connected to both the supply and exhaust \texttt{s4bldg:Fan} instances while the exhaust fan is further connected to an exhaust \texttt{Node} instance. 
Here, the \texttt{Node} class represents nodes in a flow network and keeps track of flow and temperature before one stream branches out or after multiple streams join together. In the demonstration case, there is only one space, resulting in the same flow properties before and after the \texttt{Node} instances. However, this addition is necessary in the case of multiple spaces and a more complex flow path topology. The exhaust \texttt{Node} instance is further connected to an exhaust \texttt{s4bldg:Damper} instance. The supply \texttt{s4bldg:Fan} is connected to the heating \texttt{s4bldg:Coil} instance, which is connected to a supply \texttt{Node} instance. Finally, this supply \texttt{Node} instance is connected to a supply \texttt{s4bldg:Damper} instance. 

For \autoref{fig:DT_saref}, inverse properties also exist for most of the shown properties, although these are omitted for clarity. For example, \texttt{s4bldg:isContainedIn} and \texttt{saref:isMeasuredByDevice} are inverse properties of \texttt{s4bldg:contains} and \texttt{saref:measuresProperty}. 




\subsection{Obtaining a working energy model}
With the SAREF representation in \autoref{fig:DT_saref} of the considered single-zone system, the topological context of all devices and components is defined. However, this representation cannot be directly used with the presented algorithms in \autoref{sec:topology} to simulate the system. Therefore, inputs and outputs of each model extension, as described in \autoref{sec:topology}, are first matched using the topological relationships shown in \autoref{fig:DT_saref} and the \texttt{s4syst:Connection} and \texttt{s4syst:ConnectionPoint} classes as demonstrated in \autoref{sec:topology}. 




For example, a connection between the \texttt{s4bldg:BuildingSpace} instance and the CO2 \texttt{saref:Sensor} instance is inferred through the properties \texttt{saref:measuresProperty} and \texttt{saref:isPropertyOf} of the \texttt{saref:Sensor} and \texttt{saref:CO2} instance, respectively. A connection between the \texttt{saref:Sensor} and \texttt{s4bldg:Controller} is established through the \texttt{saref:controlsProperty} and \texttt{saref:isMeasuredByDevice} properties. 



The resulting connections are shown in \autoref{fig:data_structure_cycle}. Completing this process for all \texttt{s4syst:System} instances produces the graph shown at the top of \autoref{fig:Topology_1space_1v_1h_0c}. Here, each pair of \texttt{s4syst:Connection} and \texttt{s4syst:ConnectionPoint} instances linking two models are represented as one edge labeled with the shorthand "C:"  and "CP:" representing the \texttt{outputName} and \texttt{inputName} properties, respectively. As mentioned previously, these property names correspond to the mathematical notation of input and outputs in \autoref{tab:model_overview}. In addition to the already discussed \texttt{Node} class, two other helper classes \texttt{OutdoorEnvironment}, and \texttt{Schedule} have also been introduced. \texttt{OutdoorEnvironment} represents relevant outdoor conditions such as outdoor temperature and solar irradiation, which affects different systems, in this case, the \texttt{s4bldg:BuildingSpace} and \texttt{s4bldg:AirToAirHeatRecovery} instances. \texttt{Schedule} is a generic class that outputs predetermined schedules for a specific period, e.g. daily occupancy or setpoint profiles. 



With the inputs and outputs defined for all components in the system, the three algorithms from \autoref{sec:topology} can be applied to simulate the system. An overview of this workflow is shown in \autoref{fig:Topology_1space_1v_1h_0c}. As a first step, \autoref{alg:remove_cycles} is applied to the set of components $H$, to identify and remove cycles caused by the controllers. This results in the updated set $H^*$, where four edges caused by the two controllers are removed. As shown in \autoref{fig:Topology_1space_1v_1h_0c}, all four removed edges point to the \texttt{s4bldg:BuildingSpace} instance, which holds the controlled properties. The obtained set of components $H^*$ is then used as input for \autoref{alg:topological_sorting} from which a sequence $L$ is obtained, containing the components of $H$ in a topologically sorted order. This sequence is then used in \autoref{alg:simulation} for simulating the original system $H$. To run this simulation, parameters have been specified for each model as listed in \autoref{tab:parameters}. The parameters are based on a classroom in an actual university building. This is the case for the temperature prediction model for the \texttt{s4bldg:BuildingSpace} instance, which was trained on actual data collected from the classroom as described in our previous work \cite{BSABjoernskov2022,BSOBjoernskov2022}. For other components such as the \texttt{s4bldg:Fan} and \texttt{s4bldg:AirToAirHeatRecovery} instances, standard values, suitable for the demonstration case, are used. It should be stressed that the simulation is not intended to validate the accuracy of the presented component models, but rather to provide a proof-of-concept and validation of the overall modeling framework, information carryover, algorithms, and workflow.   


\subsection{Simulation results}

For simulation, a 4-day winter period using actual historical weather data from Denmark is considered. The results for some of the components are shown in \autoref{fig:simulation} where the output of each model is shown with the black dashed line while the colored lines show model inputs. The weather data used for the simulation are shown in \autoref{fig:outdoor_environment} as a separate plot, characterized by low solar irradiance and temperatures.



Results for the temperature and CO2 models attached to the \texttt{s4bldg:BuildingSpace} instance are shown in the two top plots, \autoref{fig:space_temperature}, and \autoref{fig:space_CO2}. \autoref{fig:temperature_controller} shows the inputs and outputs of the temperature \texttt{s4bldg:Controller} instance. As shown, a constant 23 $^\circ$C setpoint $T_{z,set}$ is provided as input to the \texttt{s4bldg:Controller} model during the whole period, resulting in a fluctuating valve position, which closes when the indoor temperature $T_z$ exceeds $T_{z,set}$ and opens when the temperature drops below $T_{z,set}$. From \autoref{fig:space_heater}, the resulting heat consumption $\dot{Q}_h$ is shown for the \texttt{s4bldg:SpaceHeater} instance following the same fluctuating pattern, based on the water massflow $\dot{m}_w$. 

Due to the low solar irradiation in the winter period, the shading position $u_{s}$ has a limited effect on the indoor temperature and is therefore set to 0 during the entire period. However, as expected, running simulations for summer periods, it is observed that shade position has a high impact on indoor temperature, due to the higher irradiation levels.


\autoref{fig:space_CO2} shows the simulation results for the CO$_2$ model attached to the \texttt{s4bldg:BuildingSpace} instance, where an artificially generated occupancy profile $N_{occ}$ is used as input. Here, $\dot{m}_a$ represents both the supply and exhaust airflow as they are equal. \autoref{fig:CO2_controller} shows the CO$_2$ setpoint $C_{z,set}$ is set constant at 600 ppm, which the controller tracks by inversely actuating the supply and exhaust damper positions $u_d$. As shown in \autoref{fig:space_CO2}, the predicted CO$_2$-concentration $C_z$ is sensitive to rapid changes in occupancy and airflow, causing the tracking error to occasionally approach  100 ppm during occupancy periods. 

\begin{figure*}
    \centering
    \begin{subfigure}{0.45\linewidth}
        \centering
        \includegraphics[width=\textwidth]{plot_space_temperature.png}
        \vspace*{\distanceToCaption mm}
        \caption{}
        \label{fig:space_temperature}
    \end{subfigure}
    \begin{subfigure}{0.45\linewidth}
        \centering
        \includegraphics[width=\textwidth]{plot_space_CO2.png}
        \vspace*{\distanceToCaption mm}
        \caption{}
        \label{fig:space_CO2}
    \end{subfigure}
    
     
    \begin{subfigure}{0.45\linewidth}
        \centering
        \includegraphics[width=\textwidth]{plot_temperature_controller.png}
        \vspace*{\distanceToCaption mm}
        \caption{}
        \label{fig:temperature_controller}
    \end{subfigure}
    \begin{subfigure}{0.45\linewidth}
        \centering
        \includegraphics[width=\textwidth]{plot_CO2_controller.png}
        \vspace*{\distanceToCaption mm}
        \caption{}
        \label{fig:CO2_controller}
    \end{subfigure}


    \begin{subfigure}{0.45\linewidth}
        \centering
        \includegraphics[width=\textwidth]{plot_space_heater.png}
        \vspace*{\distanceToCaption mm}
        \caption{}
        \label{fig:space_heater}
    \end{subfigure}
    \begin{subfigure}{0.45\linewidth}
        \centering
        \includegraphics[width=\textwidth]{plot_supply_fan.png}
        \vspace*{\distanceToCaption mm}
        \caption{}
        \label{fig:supply_fan}
    \end{subfigure}
    
    \begin{subfigure}{0.45\linewidth}
        \centering
        \includegraphics[width=\textwidth]{plot_air_to_air_heat_recovery.png}
        \vspace*{\distanceToCaption mm}
        \caption{}
        \label{fig:air_to_air_heat_recovery}
    \end{subfigure}
    \begin{subfigure}{0.45\linewidth}
        \centering
        \includegraphics[width=\textwidth]{plot_heating_coil.png}
        \vspace*{\distanceToCaption mm}
        \caption{}
        \label{fig:heating_coil}
    \end{subfigure}
     
    \begin{subfigure}{0.45\linewidth}
        \centering
        \includegraphics[width=\textwidth]{plot_outdoor_environment.png}
        \vspace*{\distanceToCaption mm}
        \caption{}
        \label{fig:outdoor_environment}
    \end{subfigure}
    \caption{Results of a 4-day winter period simulation for the demonstration case, employing the presented component models and modeling framework.}
    \label{fig:simulation}
\end{figure*}

The predicted power consumption $\dot{W}_{fan}$ for the supply \texttt{s4bldg:Fan} instance is shown in \autoref{fig:supply_fan}. As expected, this consumption closely follows the supply airflow $\dot{m}_a$. The exhaust fan power consumption is identical due to identical models and balanced supply and exhaust airflows and is therefore not shown. 

\autoref{fig:air_to_air_heat_recovery} show the predicted outlet temperature at the supply side $T_{a,sup,out}$ of the \texttt{s4bldg:AirToAirHeatExchanger} instance. The inputs here are inlet temperature $T_{a,sup,in}$, i.e. the outdoor temperature, the exhaust inlet temperature $T_{a,exh,in}$, i.e. the indoor air temperature predicted by the \texttt{s4bldg:BuildingSpace} model, and the airflows at the supply and exhaust side $\dot{m}_a$. 

The simulated heat consumption for the heating \texttt{s4bldg:Coil} instance $\dot{Q}_{hc}$ is shown in \autoref{fig:heating_coil}. The consumption closely follows the trend of the supply airflow $\dot{m}_a$, due to a constant setpoint $T_{a,set}$ and a relatively invariant inlet air temperature $T_{a,in}$ between 18-20 $^\circ$C. 



\subsection{Discussion}

As demonstrated from the generated results, the modeling framework enables direct dynamic simulation of the components and systems represented through SAREF semantics. Although the demonstration takes basis in a simple single-zone system, the framework is not case-specific and can readily be applied to larger systems.

The majority of the presented component models are validated through other studies and implemented in simulation tools. However, their performance and effectiveness, as part of the presented modeling framework and a dynamic DT environment, are important to establish. Implementation of the framework in a real building case study is therefore planned as part of the next steps. Here, model parameters should be specified through a combination of available design data, and parameter estimation using measured operational data. The semantic alignment between SAREF4BLDG and IFC potentially provides an advantage in the task of obtaining the required design information and properties of the modeled devices and systems. 

In addition, given the fact that sensor and meter equipment is virtually represented through the \texttt{saref:Sensor} and \texttt{saref:Meter} classes, the task of retrieving and comparing actual readings from the physical system and simulated readings from the virtual system for the same device is conceptually straightforward, which facilitates the implementation of continuous commissioning and parameter estimation. Relating to the demonstration case, the virtual temperature \texttt{saref:Sensor} instance reads the simulated indoor temperature shown in \autoref{fig:space_temperature} while the CO$_2$  \texttt{saref:Sensor} instance reads the simulated CO$_2$-concentration shown in \autoref{fig:space_CO2}. These readings are logged as \texttt{saref:Measurement} instances and retrieved through the \texttt{saref:hasFunction} and \texttt{saref:hasSensingRange} properties. 
Therefore, if actual readings from the two physical sensors are represented using the same concepts, the primary communication point from the physical system to the virtual system would thus pass through the physical and virtual \texttt{saref:Sensor} and \texttt{saref:Meter} instances. Here, robust and automated live communication with the physical IoT devices and preprocessing of the data will play a crucial role in the overall DT solution for proper use in continuous commissioning and parameter estimation. 


















% \section{Results and Discussion}
\label{sec:results_and_discussion}

\def\distanceToCaption{-7}
\begin{figure*}
     \centering
     \begin{subfigure}{0.45\linewidth}
         \centering
         \includegraphics[width=\textwidth]{plot_space_temperature.png}
         \vspace*{\distanceToCaption mm}
         \caption{}
         \label{fig:}
     \end{subfigure}
     \begin{subfigure}{0.45\linewidth}
         \centering
         \includegraphics[width=\textwidth]{plot_space_CO2.png}
         \vspace*{\distanceToCaption mm}
         \caption{}
         \label{fig:}
     \end{subfigure}
     
     \begin{subfigure}{0.45\linewidth}
         \centering
         \includegraphics[width=\textwidth]{plot_space_heater.png}
         \vspace*{\distanceToCaption mm}
         \caption{}
         \label{fig:}
     \end{subfigure}
          \begin{subfigure}{0.45\linewidth}
         \centering
         \includegraphics[width=\textwidth]{plot_supply_fan.png}
         \vspace*{\distanceToCaption mm}
         \caption{}
         \label{fig:}
     \end{subfigure}
     
     \begin{subfigure}{0.45\linewidth}
         \centering
         \includegraphics[width=\textwidth]{plot_temperature_controller.png}
         \vspace*{\distanceToCaption mm}
         \caption{}
         \label{fig:}
     \end{subfigure}
     \begin{subfigure}{0.45\linewidth}
         \centering
         \includegraphics[width=\textwidth]{plot_CO2_controller.png}
         \vspace*{\distanceToCaption mm}
         \caption{}
         \label{fig:}
     \end{subfigure}
     
     \begin{subfigure}{0.45\linewidth}
         \centering
         \includegraphics[width=\textwidth]{plot_heating_coil.png}
         \vspace*{\distanceToCaption mm}
         \caption{}
         \label{fig:}
     \end{subfigure}
     \begin{subfigure}{0.45\linewidth}
         \centering
         \includegraphics[width=\textwidth]{plot_heat_recovery_unit.png}
         \vspace*{\distanceToCaption mm}
         \caption{}
         \label{fig:}
     \end{subfigure}
     
     \begin{subfigure}{0.45\linewidth}
         \centering
         \includegraphics[width=\textwidth]{plot_weather_station.png}
         \vspace*{\distanceToCaption mm}
         \caption{}
         \label{fig:}
     \end{subfigure}
        \caption{X-day simulation for the system shown in}
        \label{fig:three graphs}
\end{figure*}


% The list of presented components is not exhaustive and could be expanded based on experiences from case studies. Similarly the presented mathematical models could also be replaced or expanded. 


% In \autoref{sec:simulation_framework} it was assumed that an appropriate data model was already available with required topology and design information. 


% \cite{BELLAGARDA2022104314} - tansfer learning gives improved performance. However, it is based on synthetic data, i.e. development of an EnergyPlus model is necessary. The already discussed drawbacks and challenges in regards to model development and automation could therefore be inhibiting.


\section{Conclusion and Future Work}

As part of the global energy transition, increasing the smartness of buildings is a key milestone, improving automation, efficiency, and flexibility. As part of this transformation, increased availability, sharing, management, and use of data from IoT devices and sensor networks both entail new opportunities as well as challenges. 

Inspired by the demonstrated success and applications of the DT concept in other fields and industries undergoing a similar digital transformation as the building sector, the building DT is now emerging as a promising solution to handle and utilize these data streams for different important services such as fault detection, operational optimization, and strategy planning. 

One of the fundamental constituents of a building DT which enables this is a dynamic simulation model. However, to avoid manual and time-consuming modeling workflows, and to ensure the adaptability and scalability of building DTs, close integration between the simulation models and IoT-devices at the building site is essential. 

This work presents an innovative energy modeling framework that builds directly upon the semantics of the SAREF ontology with an emphasis on modular data-driven models. With a basis in the SAREF4BLDG extension, different candidate models were considered as extensions for a selection of classes. Using the existing ontology patterns from the SAREF4SYST extension, a generic method for systematically linking model inputs and outputs between components was presented. To enable direct dynamic simulation of the different components and systems, three essential algorithms were then presented. As a proof-of-concept, the framework was applied to model and simulate a generic single-zone system with common building components and systems. 
The presented framework is considered the foundation for future work that will consider an actual building as a case study with the integration of actual devices, real-time sensor equipment, parameter estimation, and delivery of different services. This will demonstrate the feasibility of using the DT services to enhance the efficiency and smartness of buildings.

The use of SAREF ontology as a semantic backbone for the modeling framework improves interoperability and integration with the actual physical system and devices and avoids translation to simulation software. This potentially enables broader adoption and scalability of building DTs. In addition, using the multiple domains covered by the SAREF extensions such as industry, energy, water, smart cities, etc., the DT concept could potentially be expanded to consider streets, districts, and cities. 



% As such the 


% The presented framework avoids conversion from ontology to building simulation software, attempting to increase interoperability and flexibility. 



% Using the proposed modeling framework, the static SAREF components are thus brought to life using dynamic models and broadly applicable concepts. 


% increases interoperability. Furthermore, the ability to reason directly between components in the simulation model using the different topology patterns could also yield valuable insights and provide knowledge discovery. 


% The presented framework is considered the foundation for future work considering with exparameter estimation and model training as well. 



% one note here is to highlight the path towards scalability in terms of using the twin from the level of a building to set of buildings and larger context...also you could open the door here to talk about the future work in implementing this in real case buildings and demonstrate the feasibility of using the DT services to enhance the efficiency and smartness of buildings..



\section*{Declaration of competing interest}

The authors declare that they have no known competing financial interests or personal relationships that could have appeared to influence the work reported in this paper.
\section*{Acknowledgements}


This work is carried out under the ‘Twin4Build: A holistic Digital Twin platform for decision-making support over the whole building life cycle’ project, funded by the Danish Energy Agency under the Energy Technology Development and Demonstration Program (EUDP), ID number: 64021-1009.

\section*{Author statement}

\textbf{Jakob Bjørnskov}: Conceptualization, Methodology, Software, Formal analysis, Investigation, Writing - original draft, Writing - review \& editing, Visualization, Project administration. \textbf{Muhyiddine Jradi}: Conceptualization, Methodology, Writing - review \& editing, Supervision, Funding acquisition.

\appendix
\section{}
\label{sec:appendix_A}



\subsection{Valve}
The valve model is based on the model used by Zajic et al. \cite{zajic2011a}, and is given by the equations \ref{eq:valve_u}-\ref{eq:valve_m_w}. 

\begin{equation}
    \tilde{u}_v = \frac{u_v}{\sqrt{u_v^2 \; (1-N_v)+N_v}}
    \label{eq:valve_u}
\end{equation}

\begin{equation}
    \dot{m}_w = \tilde{u} \; m_{w,max}
    \label{eq:valve_m_w}
\end{equation}

In \autoref{eq:valve_u}, the effectively applied stem position $\tilde{u}$ is calculated based on the control input $u \in [0,1]$, and the valve authority $N_v \in [0,1]$. Following, the massflow can then be calculated in \autoref{eq:valve_m_w} using the maximum massflow $\dot{m}_{w,max}$. 

\subsection{Space heater}
\label{sec:space_heater}

The space heater model is based on the energy balance equation \ref{eq:dT}. 

% \begin{equation}
%     \bar{T}_{r} = \frac{T_{w,in} + T_{w,out}}{2}
%     \label{eq:T_avg}
% \end{equation}

\begin{equation}
    C_r \; \frac{\mathrm{d} \bar{T}_{r}}{\mathrm{d}t} = \dot{m}_w c_{p,w} (T_{w,in}-T_{w,out}) - U\!A (\bar{T}_{r}-T_z)
    \label{eq:dT}
\end{equation}


\autoref{eq:dT} represents the energy balance of the space heater with the stored heat on the left-hand side and the ingoing and outgoing heat flows on the right-hand side. Here, $\bar{T}_{r}$ is the effective temperature of the space heater, $T_{w,in}$ is the water inlet temperature, and $T_{w,out}$ is the water outlet temperature. $C_r$ is the heat capacity of the space heater, $\dot{m}_w$ is the water massflow, $c_{p,w}$ is the specific heat capacity of water, $U\!A$ is the heat transfer coefficient, and $T_z$ is the room temperature.

Xu et al. \cite{XU20081755} models the effective temperature $\bar{T}_{r}$ of a space heater as one lumped element governed by equation \ref{eq:dT} with a varying nonlinear $U\!A$ based on flow conditions. Tahersima et al. \cite{Tahersima2010ThermalAO} consider the space heater as multiple finite elements that each are governed by equation \ref{eq:dT} with a constant $U\!A$. For simplicity, this work considers the space heater as one lumped element with a constant $U\!A$ and $\bar{T}_{r}=T_{w,out}$ is assumed, i.e. all energy transferred to the room is driven by the outlet water temperature. Through Backward Euler discretization  \cite{ODE_numerical_analysis} of equation \ref{eq:dT}, an explicit expression for the effective space heater temperature $\bar{T}_{r,t}$ at time $t$ can thus be obtained as shown in \autoref{eq:T_r}.

\begin{equation}
    \bar{T}_{r,t} =  \frac{\left(T_{w,in} \dot{m}_w c_{p,w} + U\!A \; \bar{T}_{z,t-1} \right)\Delta t + \bar{T}_{r,t-1}C_r}{\left(U\!A + \dot{m}_w c_{p,w}\right)\Delta t + C_r}
    \label{eq:T_r}
\end{equation}

Here, $\Delta t$ is the timestep and $\bar{T}_{r,t-1}$ is the effective space heater temperature from the previous timestep. Based on the obtained space heater temperature $\bar{T}_{r,t}$, the heat transferred to the space heater from the heating system can be calculated in \autoref{eq:Q_h}. 

% \begin{equation}
%     T_{w,out} = 2\bar{T}_{r,t}-T_{w,in}
%     \label{eq:T_w_out}
% \end{equation}

\begin{equation}
    \dot{Q}_h = \dot{m}_w c_{p,w} (T_{w,in}-T_{w,out})
    \label{eq:Q_h}
\end{equation}



\subsection{Heating coil}
The heating coil model is based on a simple heat balance equation by considering the sensible heat transfer from the coil to the air and is given by \autoref{eq:heating_coil}. 

\begin{equation}
    \dot{Q}_{hc} = \begin{cases}
        \dot{m}_a c_{p,a} (T_{a,set}-T_{a,in}),& \text{if } T_{a,in}<T_{a,set}\\
        0,              & \text{otherwise}
    \end{cases}
    \label{eq:heating_coil}
\end{equation}

In \autoref{eq:heating_coil}, $\dot{m}_a$ is the massflow of air, $c_{p,a}$ is the specific heat capacity of air, $T_{a,set}$ is the temperature setpoint of the supply airflow, and $T_{a,in}$ is the inlet air temperature of the airflow with respect to the heating coil. The model assumes that the heating coil is ideal and can always meet the specified supply air setpoint. 

\subsection{Cooling coil}
Similarly to the heating coil, the cooling coil model also is based on a simple heat balance equation by considering the sensible heat transfer from the air to the coil and is given by \autoref{eq:cooling_coil}. 

\begin{equation}
    \dot{Q}_{cc} = \begin{cases}
        \dot{m}_a c_{p,a} (T_{a,in}-T_{a,set}),& \text{if } T_{a,in}>T_{a,set}\\
        0,              & \text{otherwise}
    \end{cases}
    \label{eq:cooling_coil}
\end{equation}

In \autoref{eq:cooling_coil}, $\dot{m}_a$ is the massflow of air, $c_{p,a}$ is the specific heat capacity of air, $T_{a,set}$ is the temperature setpoint of the supply airflow, and $T_{a,in}$ is the inlet air temperature of the airflow with respect to the cooling coil. 




\subsection{Air to air heat recovery}

The heat recovery model is based on the EnergyPlus \textit{Air System Air-To-Air Sensible and Latent Effectiveness Heat Exchanger model} \cite{energy2021a} and is given by equations \ref{eq:HR_f_flow}-\ref{eq:T_supairout}. 

\begin{equation}
    f_{flow} = \frac{\frac{1}{2} \left( \dot{m}_{a,sup} + \dot{m}_{a,exh} \right)}{\dot{m}_{a,max}}
    \label{eq:HR_f_flow}
\end{equation}

\begin{equation}
    \epsilon_{75\%} = \begin{cases}
        \epsilon_{75\%,h},& \text{if } T_{a,sup,in}<T_{a,exh,in}\\
        \epsilon_{75\%,c},              & \text{otherwise}
    \end{cases}
    \label{eq:eps_75}
\end{equation}

\begin{equation}
    \epsilon_{100\%} = \begin{cases}
        \epsilon_{100\%,h},& \text{if } T_{a,sup,in}<T_{a,exh,in}\\
        \epsilon_{100\%,c},              & \text{otherwise}
    \end{cases}
    \label{eq:eps_100}
\end{equation}

\begin{equation}
    \epsilon_s = \epsilon_{s,75\%} + \left( \epsilon_{s,100\%} - \epsilon_{s,75\%} \right) \frac{f_{flow} - 0.75}{1 - 0.75} 
    \label{eq:eps_s}
\end{equation}

\begin{equation}
    \dot{C}_{a,sup} = \dot{m}_{a,sup} c_{p,a}
    \label{eq:HR_Csup}
\end{equation}


\begin{equation}
    \dot{C}_{a,exh} = \dot{m}_{a,exh} c_{p,a}
    \label{eq:HR_Cexh}
\end{equation}

\begin{equation}
    \dot{C}_{a,min} = \min \left( \dot{C}_{a,sup}, \dot{C}_{a,exh} \right)
    \label{eq:HR_Cmin}
\end{equation}


\begin{equation}
    T_{a,sup,out} = T_{OA} + \epsilon_{s} \left( \frac{\dot{C}_{a,min}}{\dot{C}_{a,sup}} \right) \left( T_{a,exh,in} - T_{OA} \right)
    \label{eq:T_supairout}
\end{equation}

In \autoref{eq:HR_f_flow} the average flow fraction $f_{flow}$ is calculated based on the supply air massflow $\dot{m}_{sup}$, the exhaust air massflow $\dot{m}_{exh}$, and the maximum air massflow $\dot{m}_{max}$. 
The heat recovery model has two modes of operation, heating, and cooling, reflected by the conditional effectiveness parameters $\epsilon_{75\%,h}$, $\epsilon_{100\%,h}$,$\epsilon_{75\%,c}$, $\epsilon_{100\%,c}$ in \autoref{eq:eps_75} and \autoref{eq:eps_100}. Here, $\epsilon_{75\%}$ and $\epsilon_{100\%}$ is the efficiency at 75 \% and 100 \% airflow, while $c$ and $h$ denotes cooling and heating mode, respectively. Based on these effectiveness values, the operating efficiency $\epsilon$ is calculated in \autoref{eq:eps_s} by linear interpolation.

$\dot{C}_{sup}$ is the heat capacity rate of the supply air and $\dot{C}_{exh}$ is the heat capacity rate of the exhaust air given by \autoref{eq:HR_Csup} and \autoref{eq:HR_Cexh}, respectively. Following, the minimum heat capacity rate $\dot{C}_{min}$ is calculated in \autoref{eq:HR_Cmin}, and finally, the outlet air temperature on the supply side $T_{a,sup,out}$ can be calculated in \autoref{eq:T_supairout}, where $T_{OA}$ is the outdoor temperature. In cases where the heat exchanger can provide setpoint control, $T_{a,sup,out}$ is corrected with \autoref{eq:T_supairout_corrected} to avoid excessive heating or cooling of the air beyond the setpoint temperature.

\begin{equation}
    T_{a,sup,out} = \begin{cases}
        \min (T_{sup,out}, T_{a,set}),& \text{if } T_{a,sup,in}<T_{a,exh,in}\\
        \max (T_{sup,out}, T_{a,set}),              & \text{otherwise}
    \end{cases}
    \label{eq:T_supairout_corrected}
\end{equation}

In a previous investigation, the authors have proposed and demonstrated the use of a data-driven approach to identify the parameters $\epsilon_{75\%,h}$ and $\epsilon_{100\%,h}$ through collected temperature data by formulating a least squares optimization problem \cite{BJORNSKOV2022104277}. The resulting model provided good prediction accuracy on the outlet air temperature compared with the measured data. 


\subsection{Fan}


The fan model is based on the EnergyPlus \textit{Variable Speed Fan Model} \cite{energy2021a} and is given by equations \ref{eq:Fan_f_flow}-\ref{eq:w_fan}.



\begin{equation}
    f_{flow} = \frac{\dot{m}_a}{\dot{m}_{a,max}}
    \label{eq:Fan_f_flow}
\end{equation}

\begin{equation}
    f_{pl} = c_1 + c_2 \; f_{flow} + c_3 \; f^2_{flow} + c_4 \; f^3_{flow}
    \label{eq:f_pl}
\end{equation}

\begin{equation}
    \dot{W}_{fan} = f_{pl} \dot{W}_{fan,max}
    \label{eq:w_fan}
\end{equation}

% \begin{equation}
%     \dot{W}_{fan} = \frac{f_{pl} \; \dot{m}_{a,max}  \; \Delta P_{max}}{\eta_{tot} \; \rho_{a}}
%     \label{eq:w_fan}
% \end{equation}

In \autoref{eq:Fan_f_flow}, the flow fraction $f_{flow}$ is first calculated based on the current massflow $\dot{m}_a$ and the maximum massflow $\dot{m}_{a,max}$. Following, in \autoref{eq:f_pl}, the part load ratio $f_{pl}$ is calculated, based on the flow fraction and the power coefficients $c_1$-$c_4$, where $c_1+c_2+c_3+c_4=1$. In \autoref{eq:w_fan}, the part load ratio is then used to calculate the fan power consumption $\dot{W}_{fan}$, based on the power consumption at maximum airflow $\dot{W}_{fan,max}$. 

\subsection{Damper}


The damper model describes the airflow through the damper $\dot{m}_a$ as a function of the damper position $u$ and the parameters $a$, $b$, and $c$. It is based on the model presented by Huang \cite{huang2011a} and is given by equation \autoref{eq:m_a}. 

\begin{equation}
    \dot{m}_{a} = a \; \mathrm{e}^{b \; u_d} + c
    \label{eq:m_a}
\end{equation}

Constraints can be imposed on the parameters $a$, $b$, $c$ in equation \autoref{eq:m_a} such that $m_a=0$ when $u_d=0$ and $m_a=m_{a,max}$ when $u_d=1$. Hence, if $a$ is given, $b$ and $c$ are given by Equations \ref{eq:c}-\ref{eq:b} to satisfy these two constraints.

\begin{equation}
    c = -a
    \label{eq:c}
\end{equation}

\begin{equation}
    b = \ln\Bigg(\frac{\dot{m}_{a,max}-c}{a}\Bigg)
    \label{eq:b}
\end{equation}

\subsection{Controller}
Properly designed and implemented Building Management Systems (BMS) are vital for the building to adapt and react effectively to internal and external dynamic changes such as occupancy and climatic conditions \cite{PANTAZARAS2016774}. The BMS is typically connected with a set of controllers that control the operation of various building systems. Controllers can come in different forms, e.g. rule-based control, conventional feedback control, predictive control, etc. However, the general aim of controllers is to modulate system inputs such that a predefined goal is achieved. For conventional controllers, the goal is typically to drive a specific system property toward the desired setpoint value. 
One of the most fundamental and widely used controllers is the Proportional Integral Derivative (PID) controller \cite{Belic2015}. In discrete time, the controller behavior is described by Equations \ref{eq:e_t}-\ref{eq:u_t} \cite{Veeranna2010}.

\begin{equation}
   e_t = y_{set}-y_{meas,t}
   \label{eq:e_t}
\end{equation}

\begin{equation}
   u_t = u_{t-1} + K_p (e_t-e_{t-1}) + K_i \Delta t +  \frac{K_d}{\Delta t} \left(e_t -2e_{t-1} + e_{t-2} \right)
   \label{eq:u_t}
\end{equation}


Where $e_t$ is the residual between the setpoint $y_{set}$ and the measured value $y_{meas,t}$ at time $t$. In \autoref{eq:u_t}, the output signal of the controller $u_t$ is calculated based on the previous signal $u_{t-1}$, the previous residuals $e_{t-1}$, $e_{t-2}$, the parameters $K_p$, $K_i$, $K_d$, and the timestep $\Delta t$. Changing the type of controller is simply a matter of adjusting the parameters $K_p$, $K_i$, $K_d$, i.e., a P controller can be modeled with $K_p \neq 0$ and $K_i$, $K_d=0$ while a PI controller can be modeled with $K_p$, $K_i \neq 0$ and $K_d=0$. Additionally, a reverse-acting PID controller can be modeled with $K_p$, $K_i$, $K_d<0$. 

% A PI controller was used by Tahersima et al. \cite{Tahersima2010ThermalAO} for controlling the room temperature in a simulation environment by using a finite element radiator model similar to the model presented in \autoref{sec:space_heater}.  



\subsection{Building Space}

Indoor comfort is the driving force of essentially all energy use in buildings. Most BMS have automated control of indoor temperature to follow specified setpoint schedules. Additionally, Demand Controlled Ventilation (DCV) is a common strategy to ensure that mechanical ventilation is only operating during occupancy presence in the building with the aim of lowering power consumption and ensuring proper Indoor Air Quality (IAQ). For this purpose, CO$_2$ concentration is predominately measured as an indicator of IAQ and is controlled to follow certain setpoint schedules, by modulating the ventilation airflows \cite{MEREMA2018349}. Therefore, the task of accurately modeling and predicting temperature and CO$_2$ concentration is essential to properly represent the actual building space in the building DT. Therefore, the space component must be capable of predicting both the dynamic temperature and CO$_2$ concentration response, depending on the operation of the HVAC system, weather conditions, and occupancy use. The dynamic CO$_2$ concentration in rooms is typically modeled with the fundamental mass balance given by \autoref{eq:dC} \cite{Macarulla2017,PANTAZARAS2016774}.

\begin{equation}
    m_z \frac{\mathrm{d} C_z}{\mathrm{d} t} = C_{sup}\dot{m}_{a,sup} - C_z\dot{m}_{a,exh} + K_{occ} N_{occ}
    \label{eq:dC}
\end{equation}

Where, $m_z$ is the mass of the air contained in the room, $C_z$ is the room CO$_2$ concentration, and $C_{sup}$ is the CO$_2$-concentration of the supply airflow, which can be assumed equal to the outdoor air CO$_2$-concentration level at 400 ppm \cite{PANTAZARAS2016774,CIBSE}. $K_{occ}$ is the rate of CO$_2$ mass generated per occupant and $N_{occ}$ is the number of occupants in the room. Pantazaras et al. \cite{PANTAZARAS2016774}, estimated $K_{occ}$, while synthetic EnergyPlus data was used as input for air flows and occupancy. Macarulla et al. \cite{Macarulla2017}, extended \autoref{eq:dC} to form a stochastic differential equation. Different configurations were tested for the estimated parameters. However, for the best-performing model, the actual occupancy count was used as input, while $K_{occ}$ and the air flows were estimated as constants. 

However, similarly to the transformation from \autoref{eq:dT} to \autoref{eq:T_r}, an explicit expression can be obtained for the room CO$_2$-concentration with Backward Euler discretization of \autoref{eq:dC}. The obtained expression is given by \autoref{eq:C_z}, where the timestep $\Delta t$ has been introduced.


\begin{equation}
    C_{z,t} = \frac{m_z C_{z,t-1} + C_{sup} \dot{m}_{a,sup} \Delta t + K_{occ}N_{occ} \Delta t}{m_z + \dot{m}_{a,exh} \Delta t}
    \label{eq:C_z}
\end{equation}

Indoor temperature forecasting has been studied extensively using different modeling methods covering both white-box, grey-box, and black-box approaches. However, recently, black-box approaches in the form of ANNs have become increasingly popular. Specifically, to deal with the transient and dynamic nature of indoor temperature, a specific branch of ANNs called Recurrent Neural Networks (RNN) is typically used. Under this category, the Long Short-Term Memory (LSTM) architecture is often used, due to its gated structure, which rectifies certain undesirable traits of its predecessor, the vanilla RNN \cite{SHERSTINSKY2020132306}. The adaptability and prediction accuracy of LSTM networks have been demonstrated across many disciplines \cite{Houdt2020}, including for the task of indoor temperature forecasting \cite{Mtibaa2020,FANG2021111053}. 

We have previously developed and demonstrated the use of the LSTM architecture \cite{BSABjoernskov2022, BSOBjoernskov2022} for indoor temperature modeling of a large set of spaces in a case study building. Therefore, this black-box approach is also applied in this work, which also demonstrates the flexibility of the proposed framework and the ability to combine different modeling approaches in a unified manner. The overall model architecture is summarized by Equations \ref{eq:X_A}-\ref{eq:dT_z}, employing two sequential LSTM networks $A$ and $B$.  

\begin{equation}
    \mathcal{X}_{A,t-1} = \big(T_{z,t-1}, T_{o,t-1}, \Phi_{s,t-1}, u_{v,t-1}, u_{d,t-1}, u_{s,t-1}\big)
    \label{eq:X_A}
\end{equation}

\begin{equation}
    (c_{A,t}, h_{A,t}) = \text{LSTM}_A\big( \mathcal{X}_{A,t-1}, c_{A,t-1}, h_{A,t-1} \big)
    \label{eq:LSTM_A}
\end{equation}

\begin{equation}
    \mathcal{X}_{B,t-1} = h_{A,t}
    \label{eq:X_B}
\end{equation}

\begin{equation}
    (c_{B,t}, h_{B,t}) = \text{LSTM}_B\big( \mathcal{X}_{B,t-1}, c_{B,t-1}, h_{B,t-1} \big)
    \label{eq:LSTM_B}
\end{equation}

\begin{equation}
    \Delta T_{z,t} = h_{B,t}
    \label{eq:dT_z}
\end{equation}



\autoref{eq:X_A} defines the input $\mathcal{X}_{A}$ for LSTM$_A$. In this case, the inputs include indoor temperature $T_z$, outdoor temperature $T_o$, shortwave irradiation $\Phi_s$, space heater valve position $u_v$, damper opening position $u_d$, and shades opening position $u_{sh}$. These inputs were chosen considering the thermal heat balance of the modeled spaces, and commonly available data in buildings. However, the inputs should be modified based on available sensors and equipment installed for the modeled space. 

In \autoref{eq:LSTM_A}, the input $\mathcal{X}_{A,t-1}$ as well as the state vectors $c_{A,t-1}$ and $h_{A,t-1}$ are provided as input for LSTM$_A$, which computes the updated state vectors $c_{A,t}$ and $h_{A,t}$. \autoref{eq:X_B} then defines the input $\mathcal{X}_{B,t-1}$ for LSTM$_B$ as the updated hidden state $h_{A,t}$. Using this input $\mathcal{X}_{B,t-1}$ as well as the state vectors $c_{B,t-1}$ and $h_{B,t-1}$, \autoref{eq:LSTM_B} then computes the updated state vectors $c_{B,t}$ and $h_{B,t}$, where the hidden state $h_{B,t}$ is treated as prediction target $\Delta T_{z,t}$, as shown in \autoref{eq:dT_z}. Here, $\Delta T_{z,t}$ is the indoor air temperature difference between the previous and current timestep. After training, the model is used in inference for indoor temperature forecasting through a closed-loop configuration, given by \autoref{eq:T_z}. Here, the indoor temperature of the next timestep is obtained by simply adding the predicted temperature change $\Delta T_{z,t}$ with the known indoor temperature of the previous timestep $T_{z,t-1}$. The newly obtained temperature $T_{z,t}$ can then be fed back as input in \autoref{eq:X_A}, closing the loop. 



\begin{equation}
    T_{z,t} = T_{z,t-1} + \Delta T_{z,t}
    \label{eq:T_z}
\end{equation}

% \begin{figure}
%      \centering
%      \begin{subfigure}[b]{1\linewidth}
%          \centering
%          \includegraphics[width=\linewidth, trim={0cm 2cm 6cm 5cm}, clip]{open_loop_architecture.png}
%          \caption{}
%          \label{fig:}
%      \end{subfigure}
%      \begin{subfigure}[b]{1\linewidth}
%          \centering
%          \includegraphics[width=\linewidth, trim={0cm 0cm 6cm 5cm}, clip]{closed_loop_architecture.png}
%          \caption{}
%          \label{fig:}
%      \end{subfigure}
%      \hfill
     
%     \caption{}
%     \label{fig:}
% \end{figure}


% such indoor temperature models for numerous spaces in a case study building \cite{BSABjoernskov2022, BSOBjoernskov2022}. Therefore, we adopt the same approach in this work. 


% An adaptive model could be implemented as demonstrated by Ruano et al. \cite{RUANO2006682}.

Although the two presented models for predicting CO$_2$ concentration and indoor temperature are both attached to the \texttt{s4bldg:BuildingSpace} component, they are essentially independent of each other.




% \begin{figure}[h]
%     \centering
%     \includegraphics[width=1\linewidth, trim={6cm 2.5cm 6.5cm 2cm}, clip]{BEM framework space.png}
%     \caption{}
%     \label{fig:space_model}
% \end{figure}

\newpage
\section{}
\begin{table}[h!]
\caption{Overview of parameter and constant values used for the demonstration case.}
\label{tab:parameters}
\resizebox*{\linewidth}{!}{%
    
    % Please add the following required packages to your document preamble:
% \usepackage{multirow}
\begin{tabular}{l|l|l}
\toprule
\textbf{Component}                        & \textbf{Parameters} & \textbf{Constants} \\ \midrule
\multirow{2}{*}{\textbf{Valve}}           & $N_v=0.8$     &                    \\
                                          & $\dot{m}_{w,max} = 0.21$ kg/s   &                    \\ \midrule
\multirow{3}{*}{\textbf{Space heater}}    & $U\!A=162$    W/K          & $c_{p,w}=4180$ J/kg/K         \\
                                          & $C_r=125000$  J/K          &   $\Delta t=600$ s                 \\
                                          &                     &                    \\ \midrule
\multirow{3}{*}{\textbf{Heating coil}}    &                     &    $c_{p,a}=1000$ J/kg/K               \\
                                          &                     &                    \\
                                          &                     &                    \\ \midrule
\multirow{5}{*}{\textbf{Air to air heat recovery} $^{[1]}$} & $\epsilon_{75\%,h}=0.79$ &  $c_{p,a}=1000$ J/kg/K         \\
                                          & $\epsilon_{75\%,c}=0.79$ &                    \\
                                          & $\epsilon_{100\%,h}=0.73$ &                    \\
                                          & $\epsilon_{100\%,c}=0.73$&                    \\
                                          & $m_{a,max}=1.7$ kg/s         &                    \\ \midrule
\multirow{7}{*}{\textbf{Supply Fan}$^{[2]}$}             & $c_1=0.027828$     &       $\rho_a=1.225$ kg/m$^3$             \\
                                          & $c_2=0.026583$     &                    \\
                                          & $c_3=-0.087069$     &                    \\
                                          & $c_4=1.030920$     &                    \\
                                          & $\dot{m}_{a,max}=1.7$ kg/s   &                    \\
                                          & $\dot{W}_{fan,max}=1500$ W    &                    \\ \midrule
\multirow{2}{*}{\textbf{Supply Damper}$^{[3]}$}          &       $a=5$           &                    \\
                                          &       $\dot{m}_{a,max}=1.63$ kg/s           &                    \\\midrule
\multirow{2}{*}{\textbf{Exhaust Damper}$^{[3]}$}          &       $a=5$           &                    \\
                                          &       $\dot{m}_{a,max}=1.63$ kg/s          &                    \\\midrule
\multirow{3}{*}{\textbf{Temperature controller}}      &      $K_p=0.05$          &                    \\ 
                                          &      $K_i=0.8$          &                    \\ 
                                          &      $K_d=0$          &                    \\ \midrule
\multirow{3}{*}{\textbf{CO2 controller}}      &      $K_p=-0.001$          &                    \\ 
                                          &      $K_i=-0.001$          &                    \\ 
                                          &      $K_d=0$          &                    \\ \midrule
\multirow{9}{*}{\textbf{Space}$^{[4]}$}  &      $K_{occ}=8.316 \cdot 10^{-6}$ kg/s/person     &        $\Delta t=600$ s  \\ 
                                          &      $m_z=571.5$ kg          &                   \\ 
                                          &                     &                    \\ 
                                          &                     &                    \\ 
                                          &         [-]         &                    \\ 
                                          &                          &                    \\ 
                                          &                          &                    \\ 
                                          &                          &                    \\ 
                                          &                          &                    \\ 
\bottomrule
\end{tabular}
}
{\footnotesize \raggedright $^{[1]}$ Efficiency values $\epsilon_{75\%,h}$, $\epsilon_{75\%,c}$, $\epsilon_{100\%,h}$, $\epsilon_{100\%,c}$ was selected based on the AHRI directory of certified product performance for a specific product \cite{AHRI}. \par}
{\footnotesize \raggedright $^{[2]}$ Power coefficients $c_1$-$c_4$ was selected based on the single-zone default values provided by the ANSI/ASHRAE/IES standard 90.1 \cite{Goel2016ANSIASHRAEIESS9}. \par}
{\footnotesize \raggedright $^{[3]}$  \autoref{eq:c} and \autoref{eq:b} are used to calculate $c$ and $b$. \par}
{\footnotesize \raggedright $^{[4]}$ CO$_2$-generation per occupant $K_{occ}$ obtained for a classroom from \cite{Persily2017}. \par}
{\footnotesize \raggedright The LSTM model contains too many parameters to show. \par}
\end{table}


\begin{table*}[h!]
    \centering
    \caption{A subset of class descriptions and related properties from the SAREF4BLDG ontology extension for components commonly considered in building energy modeling \cite{saref4bldg}.}
    
\newcommand*\documentclassCustomStyleList{final,5p,times,twocolumn}
\newcommand\documentclassCustomCommand[1][]{\expandafter\documentclass\expandafter[#1]}
\documentclassCustomCommand[class=elsarticle,\documentclassCustomStyleList]{standalone}
\usepackage{temp_style}



\begin{document}
% \begin{table}[h]
% \caption{Overview of inputs, outputs, parameters, and constants of potential mathematical models for the highlighted components.}
% \label{tab:}
\resizebox{0.9\linewidth}{!}{%
\begin{tabular}{lll}
\toprule
\textbf{Component}                & \textbf{Description}                                                                                   & \textbf{Properties}         \\ \midrule
\textbf{Valve}                    & \textit{A valve is used in a building services piping distribution system}                                      & \texttt{closeOffRating}$^\textrm{op}$            \\
                                  & \textit{to control or modulate the flow of the fluid.}                                                          & \texttt{flowCoefficient}$^\textrm{op}$           \\
                                  &                                                                                                        & \texttt{size}$^\textrm{op}$                      \\
                                  &                                                                                                        & \texttt{testPressure}$^\textrm{op}$              \\
                                  &                                                                                                        & \texttt{valveMechanism}$\color{DarkGreen}^\textrm{dp}$            \\
                                  &                                                                                                        & \texttt{valveOperation}$\color{DarkGreen}^\textrm{dp}$            \\
                                  &                                                                                                        & \texttt{valvePattern}$\color{DarkGreen}^\textrm{dp}$              \\
                                  &                                                                                                        & \texttt{workingPressure}$^\textrm{op}$           \\ \midrule
\textbf{Space Heater}             & \textit{Space heaters utilize a combination of radiation and/or natural convection}                             & \texttt{bodyMass}$^\textrm{op}$                  \\
                                  & \textit{using a heating source such as electricity, steam or hot water to heat a limited space or area.}        & \texttt{energySource}$\color{DarkGreen}^\textrm{dp}$              \\
                                  & \textit{Examples of space heaters include radiators, convectors, baseboard and finned-tube heaters.}            & \texttt{heatTransferDimension}$\color{DarkGreen}^\textrm{dp}$     \\
                                  &                                                                                                        & \texttt{heatTransferMedium}$\color{DarkGreen}^\textrm{dp}$        \\
                                  &                                                                                                        & \texttt{numberOfPanels}$\color{DarkGreen}^\textrm{dp}$            \\
                                  &                                                                                                        & \texttt{numberOfSections}$\color{DarkGreen}^\textrm{dp}$          \\
                                  &                                                                                                        & \texttt{outputCapacity}$^\textrm{op}$            \\
                                  &                                                                                                        & \texttt{placementType}$\color{DarkGreen}^\textrm{dp}$             \\
                                  &                                                                                                        & \texttt{temperatureClassification}$\color{DarkGreen}^\textrm{dp}$ \\
                                  &                                                                                                        & \texttt{thermalEfficiency}$^\textrm{op}$         \\
                                  &                                                                                                        & \texttt{thermalMassHeatCapacity}$^\textrm{op}$   \\ \midrule
\textbf{Coil}                     & \textit{A coil is a device used to provide heat transfer between non-mixing media.}                             & \texttt{airFlowRateMax}$^\textrm{op}$            \\
                                  & \textit{A common example is a cooling coil, which utilizes a finned coil in which circulates chilled water,}    & \texttt{airFlowRateMin}$^\textrm{op}$            \\
                                  & \textit{antifreeze, or refrigerant that is used to remove heat from air moving across the surface of the coil.} & \texttt{nominalLatentCapacity}$^\textrm{op}$     \\
                                  & \textit{A coil may be used either for heating or cooling purposes by placing a series of tubes (the coil)}      & \texttt{nominalSensibleCapacity}$^\textrm{op}$   \\
                                  & \textit{carrying a heating or cooling fluid into an airstream. The coil may be constructed from tubes}          & \texttt{nominalUa}$^\textrm{op}$                 \\
                                  & \textit{bundled in a serpentine form or from finned tubes that give a extended heat transfer surface.}          & \texttt{operationTemperatureMax}$^\textrm{op}$   \\
                                  &                                                                                                        & \texttt{operationTemperatureMin}$^\textrm{op}$   \\
                                  &                                                                                                        & \texttt{placementType}$\color{DarkGreen}^\textrm{dp}$             \\ \midrule
\textbf{Air To Air Heat Recovery} & \textit{An air-to-air heat recovery device employs a counter-flow heat exchanger}                               & \texttt{hasDefrost}$\color{DarkGreen}^\textrm{dp}$                \\
                                  & \textit{between inbound and outbound air flow. It is typically used to transfer heat}                           & \texttt{heatTransferTypeEnum}$\color{DarkGreen}^\textrm{dp}$      \\
                                  & \textit{from warmer air in one chamber to cooler air in the second chamber}                                     & \texttt{operationTemperatureMax}$^\textrm{op}$   \\
                                  & \textit{(i.e., typically used to recover heat from the conditioned air being exhausted}                         & \texttt{operationTemperatureMin}$^\textrm{op}$   \\
                                  & \textit{and the outside air being supplied to a building),}                                                     & \texttt{primaryAirFlowRateMax}$^\textrm{op}$     \\
                                  & \textit{resulting in energy savings from reduced heating (or cooling) requirements.}                            & \texttt{primaryAirFlowRateMin}$^\textrm{op}$     \\
                                  &                                                                                                        & \texttt{secondaryAirFlowRateMax}$^\textrm{op}$   \\
                                  &                                                                                                        & \texttt{secondaryAirFlowRateMin}$^\textrm{op}$   \\ \midrule
\textbf{Fan}                      & \textit{A fan is a device which imparts mechanical work on a gas.}                                              & \texttt{capacityControlType}$\color{DarkGreen}^\textrm{dp}$       \\
                                  & \textit{A typical usage of a fan is to induce airflow in a building services air distribution system.}          & \texttt{motorDriveType}$\color{DarkGreen}^\textrm{dp}$            \\
                                  &                                                                                                        & \texttt{nominalAirFlowRate}$^\textrm{op}$        \\
                                  &                                                                                                        & \texttt{nominalPowerRate}$^\textrm{op}$          \\
                                  &                                                                                                        & \texttt{nominalRotationSpeed}$^\textrm{op}$      \\
                                  &                                                                                                        & \texttt{nominalStaticPressure}$^\textrm{op}$     \\
                                  &                                                                                                        & \texttt{nominalTotalPressure}$^\textrm{op}$      \\
                                  &                                                                                                        & \texttt{operationTemperatureMax}$^\textrm{op}$   \\
                                  &                                                                                                        & \texttt{operationTemperatureMin}$^\textrm{op}$   \\
\textbf{}                         &                                                                                                        & \texttt{operationalRiterial}$^\textrm{op}$       \\ \midrule
\textbf{Damper}                   & \textit{A damper typically participates in an HVAC duct distribution system}                                    & \texttt{airFlowRateMax}$^\textrm{op}$            \\
                                  & \textit{and is used to control or modulate the flow of air.}                                                    & \texttt{bladeAction}$\color{DarkGreen}^\textrm{dp}$               \\
                                  &                                                                                                        & \texttt{bladeEdge}$\color{DarkGreen}^\textrm{dp}$                 \\
                                  &                                                                                                        & \texttt{bladeShape}$\color{DarkGreen}^\textrm{dp}$                \\
                                  &                                                                                                        & \texttt{bladeThickness}$^\textrm{op}$            \\
                                  &                                                                                                        & \texttt{closeOffRating}$^\textrm{op}$            \\
                                  &                                                                                                        & \texttt{faceArea}$^\textrm{op}$                  \\
                                  &                                                                                                        & \texttt{frameDepth}$^\textrm{op}$                \\
                                  &                                                                                                        & \texttt{frameThickness}$^\textrm{op}$            \\
                                  &                                                                                                        & \texttt{frameType}$\color{DarkGreen}^\textrm{dp}$                 \\
                                  &                                                                                                        & \texttt{leakageFullyClosed}$^\textrm{op}$        \\
                                  &                                                                                                        & \texttt{nominalAirFlowRate}$^\textrm{op}$        \\
                                  &                                                                                                        & \texttt{numberOfBlades}$\color{DarkGreen}^\textrm{dp}$            \\
                                  &                                                                                                        & \texttt{openPressureDrop}$^\textrm{op}$          \\
                                  &                                                                                                        & \texttt{operation}$\color{DarkGreen}^\textrm{dp}$                 \\
                                  &                                                                                                        & \texttt{operationTemperatureMax}$^\textrm{op}$   \\
                                  &                                                                                                        & \texttt{operationTemperatureMin}$^\textrm{op}$   \\
                                  &                                                                                                        & \texttt{orientation}$\color{DarkGreen}^\textrm{dp}$               \\
                                  &                                                                                                        & \texttt{temperatureRating}$^\textrm{op}$         \\
                                  &                                                                                                        & \texttt{workingPressureMax}$^\textrm{op}$        \\ \midrule
\textbf{Controller}               & \textit{A controller is a device that monitors inputs}                                                          &                             \\
                                  & \textit{and controls outputs within a building automation system.}                                              &                             \\
                                  & \textit{A controller may be physical (having placement within a spatial structure)}                             &                             \\
                                  & \textit{or logical (a software interface or aggregated within a programmable physical controller).}             &                             \\ \midrule
\textbf{Building Space}           & \textit{An entity used to define the physical spaces of the building.}                                          &  \texttt{contains}$^\textrm{op}$      \\
                                  & \textit{A building space contains devices or building objects.}                                                 &  \texttt{hasSpace}$^\textrm{op}$      \\
                                  &                                                                                                                 &  \texttt{isSpaceOf}$^\textrm{op}$      \\
\bottomrule

\end{tabular}
}
% \end{table}
\end{document}


    \label{tab:s4bldg_components}
\end{table*}





%% The Appendices part is started with the command \appendix;
%% appendix sections are then done as normal sections
% \appendix

% \section{Sample Appendix Section}
% \label{sec:sample:appendix}
% Lorem ipsum dolor sit amet, consectetur adipiscing elit, sed do eiusmod tempor section \ref{sec:sample1} incididunt ut labore et dolore magna aliqua. Ut enim ad minim veniam, quis nostrud exercitation ullamco laboris nisi ut aliquip ex ea commodo consequat. Duis aute irure dolor in reprehenderit in voluptate velit esse cillum dolore eu fugiat nulla pariatur. Excepteur sint occaecat cupidatat non proident, sunt in culpa qui officia deserunt mollit anim id est laborum.

%% If you have bibdatabase file and want bibtex to generate the
%% bibitems, please use
%%
\clearpage
 \bibliographystyle{elsarticle-num} 
 \bibliography{cas-refs}
 
%  \begin{figure*}[h]
% \centering
% % \captionsetup[figure]{width=0.7\linewidth}
% % \captionsetup[subfigure]{width=0.7\linewidth}

% \begin{subfigure}[t]{1\linewidth}
%   \centering
%   \includegraphics[width=1\linewidth, trim={0 0cm 0 0cm}, clip]{Closed_loop_predict_results_winter.png}
%   \caption{}
%   \label{fig:Closed_loop_predict_results_winter}
% \end{subfigure}
% \begin{subfigure}[t]{1\linewidth}
%   \centering
%   \includegraphics[width=1\linewidth, trim={0 0cm 0 0cm}, clip]{Closed_loop_predict_results_summer.png}
%   \caption{}
%   \label{fig:Closed_loop_predict_results_summer}
% \end{subfigure}

% \caption{\textbf{(a)} 24-hour temperature forecast in a winter month compared with actual measured temperature. The weather input is shown on the rightmost plot, while the individual inputs for the three selected space models are shown on the plots to the left. \textbf{(b)} 24-hour temperature forecast in a summer month compared with actual measured temperature. The weather input is shown on the rightmost plot, while the individual inputs for the three selected space models are shown on the plots to the left.}
% \label{fig:space_model_results}
% \end{figure*}


 
 
%  \begin{figure*}[h!]
%     \centering
%     \includegraphics[width=1\linewidth, trim={0 0cm 0 0cm}, clip]{BEM framework HVAC.png}
%     \caption{}
%     \label{fig:}
% \end{figure*}

%% else use the following coding to input the bibitems directly in the
%% TeX file.

% \begin{thebibliography}{00}

% %% \bibitem{label}
% %% Text of bibliographic item

% \bibitem{}

% \end{thebibliography}
\end{document}
\endinput
%%
%% End of file `elsarticle-template-num.tex'.
