\section{Conclusion and Future Work}

As part of the global energy transition, increasing the smartness of buildings is a key milestone, improving automation, efficiency, and flexibility. As part of this transformation, increased availability, sharing, management, and use of data from IoT devices and sensor networks both entail new opportunities as well as challenges. 

Inspired by the demonstrated success and applications of the DT concept in other fields and industries undergoing a similar digital transformation as the building sector, the building DT is now emerging as a promising solution to handle and utilize these data streams for different important services such as fault detection, operational optimization, and strategy planning. 

One of the fundamental constituents of a building DT which enables this is a dynamic simulation model. However, to avoid manual and time-consuming modeling workflows, and to ensure the adaptability and scalability of building DTs, close integration between the simulation models and IoT-devices at the building site is essential. 

This work presents an innovative energy modeling framework that builds directly upon the semantics of the SAREF ontology with an emphasis on modular data-driven models. With a basis in the SAREF4BLDG extension, different candidate models were considered as extensions for a selection of classes. Using the existing ontology patterns from the SAREF4SYST extension, a generic method for systematically linking model inputs and outputs between components was presented. To enable direct dynamic simulation of the different components and systems, three essential algorithms were then presented. As a proof-of-concept, the framework was applied to model and simulate a generic single-zone system with common building components and systems. 
The presented framework is considered the foundation for future work that will consider an actual building as a case study with the integration of actual devices, real-time sensor equipment, parameter estimation, and delivery of different services. This will demonstrate the feasibility of using the DT services to enhance the efficiency and smartness of buildings.

The use of SAREF ontology as a semantic backbone for the modeling framework improves interoperability and integration with the actual physical system and devices and avoids translation to simulation software. This potentially enables broader adoption and scalability of building DTs. In addition, using the multiple domains covered by the SAREF extensions such as industry, energy, water, smart cities, etc., the DT concept could potentially be expanded to consider streets, districts, and cities. 



% As such the 


% The presented framework avoids conversion from ontology to building simulation software, attempting to increase interoperability and flexibility. 



% Using the proposed modeling framework, the static SAREF components are thus brought to life using dynamic models and broadly applicable concepts. 


% increases interoperability. Furthermore, the ability to reason directly between components in the simulation model using the different topology patterns could also yield valuable insights and provide knowledge discovery. 


% The presented framework is considered the foundation for future work considering with exparameter estimation and model training as well. 



% one note here is to highlight the path towards scalability in terms of using the twin from the level of a building to set of buildings and larger context...also you could open the door here to talk about the future work in implementing this in real case buildings and demonstrate the feasibility of using the DT services to enhance the efficiency and smartness of buildings..

